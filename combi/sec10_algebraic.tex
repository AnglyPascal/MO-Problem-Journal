\newpage\section{Linear Algebra}
	
	\begin{myitemize}
		\item \href{http://yufeizhao.com/olympiad/algcomb.pdf}{Algebraic Techniques - Yufei Zhao}
	\end{myitemize}

	
	\lem{Multiplication Order}{\[(AB)^T=B^TA^T\]\[ABCD=A(BC)D\]}
	
	\theo{https://artofproblemsolving.com/community/c6h355915p2270176}{Linear map that reverses the order of multiplication}{Let $F$ be a field, and $m$ a positive integer. The only $F$-linear maps $s:\mathrm{M}_m\left(F\right)\to\mathrm{M}_m\left(F\right)$ which satisfy 
		\[s\left(X_kX_{k-1}...X_1\right) = s\left(X_1\right) s\left(X_2\right) ... s\left(X_k\right)\quad \forall k\in\N,\ X_i \in\mathrm{M}_m(F)\]
	are maps of the form $s\left(P\right) = UP^TU^{-1}\quad \forall P\in\mathrm{M}_m\left(F\right)$ for $U$ an invertible $m\times m$ matrix over $F$.}

		\proof{Let $ s $ be a map that satisfies the problem condition. Let $ t $ be another linear map defined by $ t(P)=s(P)^T\quad \forall P\in\mathrm{M}(F) $.
		
		This map $ t $ also satisfies \[t\left(X_kX_{k-1}...X_1\right) = t\left(X_1\right) t\left(X_2\right) ... t\left(X_k\right)\quad \forall k\in\N,\ X_i \in\mathrm{M}_m(F)\]
		%
		The rest uses advanced algebra :cold\textunderscore sweat: something called \texttt{Noether-Skolem theorem}... maybe later...}



	\prob{https://artofproblemsolving.com/community/c6h355915p1934456}{ISL 2009 C3}{H}{Let $ n $ be a positive integer. Given a sequence $ \varepsilon_1 $ , $ \dots $ , $ \varepsilon_{n - 1} $ with $ \varepsilon_i = 0 $ or $ \varepsilon_i = 1 $ for each $ i = 1 $ , $ \dots $ , $ n - 1 $ , the sequences $ a_0 $ , $ \dots $ , $ a_n $ and $ b_0 $ , $ \dots $ , $ b_n $ are constructed by the following rules:
		\[a_0 = b_0 = 1, \quad a_1 = b_1 = 7,\]
		\[\begin{array}{lll} a_{i+1} = \begin{cases} 2a_{i-1} + 3a_i, \\ 3a_{i-1} + a_i, \end{cases} & \begin{array}{l} \text{if } \varepsilon_i = 0, \\ \text{if } \varepsilon_i = 1, \end{array} & \text{for each } i = 1, \dots, n - 1,
		
		\\[1.5em] b_{i+1}= \begin{cases} 2b_{i-1} + 3b_i, \\ 3b_{i-1} + b_i, \end{cases} & \begin{array}{l} \text{if } \varepsilon_{n-i} = 0, \\ \text{if } \varepsilon_{n-i} = 1, \end{array} & \text{for each } i = 1, \dots, n - 1. \end{array}\]
		
		Prove that $ a_n = b_n $.}\label{problem:bijection_3}\label{problem:recursive_solution_3}
	
	\begin{solution}[Algebraic, darij grinberg]
		We'll first translate the problem in matrix language, then find a linear transformation to prove it.
		
		\begin{solution_def}
			Let $ M_2(\Q) $ be the ring of $ 2\times 2 $ matrices over the rational numbers. 
			
			Definte two matrices $ A, B \in M_2(\Q) $ by $A=\left(\begin{array}{cc} 3&2\\ 1&0\end{array}\right)$ and $B=\left(\begin{array}{cc} 1&3\\ 1&0\end{array}\right)$.
			
			For every $i\in\left\{1,2,...,n-1\right\}$, define a matrix $K_i\in \mathrm{M}_2\left(\mathbb Q\right)$ by \[K_i=\varepsilon_i B + \left(1-\varepsilon_i\right) A\]This clearly yields that $K_i=A$ if $\varepsilon_i=0$, and that $K_i=B$ if $\varepsilon_i=1$.
			
			For every $i\in\left\{1,2,...,n-1\right\}$, we have $\left(\begin{array}{c} a_{i+1}\\ a_i\end{array}\right) = K_i \left(\begin{array}{c} a_{i}\\ a_{i-1}\end{array}\right)$. By induction we have 
			\[\left(\begin{array}{c} a_{n}\\ a_{n-1}\end{array}\right) =K_{n-1}K_{n-2}...K_1\left(\begin{array}{c} 7\\ 1\end{array}\right)\]
			\[a_n=\left(\begin{array}{cc} 1\ 0 \end{array}\right) K_{n-1}K_{n-2}...K_1\left(\begin{array}{c} 7\\ 1\end{array}\right)\]
			%				
			We need to prove that
			\[\left(\begin{array}{cc} 1\ 0 \end{array}\right) K_{n-1}K_{n-2}...K_1\left(\begin{array}{c} 7\\ 1\end{array}\right) = \left(\begin{array}{cc} 1\ 0 \end{array}\right) K_1K_2...K_{n-1}\left(\begin{array}{c} 7\\ 1\end{array}\right)\]
		\end{solution_def}
		
		In order to do this, it is clearly enough to define some map $s : \mathrm{M}_2\left(\mathbb Q\right) \to \mathrm{M}_2\left(\mathbb Q\right)$ which satisfies 
		\vspace{-1em}
		\begin{enumerate}
			\itemsep-.5em
			\item $ s\left(K_{n-1}K_{n-2}...K_1\right) = K_1K_2...K_{n-1} $
			\item Every matrix $P\in \mathrm{M}_2\left(\mathbb Q\right)$ satisfies $\left(\begin{array}{cc} 1&0 \end{array}\right) P\left(\begin{array}{c} 7\\ 1\end{array}\right) = \left(\begin{array}{cc} 1\ 0 \end{array}\right) s\left(P\right)\left(\begin{array}{c} 7\\ 1\end{array}\right)$
		\end{enumerate}
		
		Let $U$ be the invertible matrix $\left(\begin{array}{cc} 7&1\\ 1&2\end{array}\right) \in \mathrm{M}_2\left(\mathbb Q\right)$. Let $s : \mathrm{M}_2\left(\mathbb Q\right) \to \mathrm{M}_2\left(\mathbb Q\right)$ be the map defined by \[s\left(P\right) = UP^TU^{-1},\quad \forall P\in M_2(\Q)\]
		
		It's easy to check with computation that this mapping satisfies (2). We need to show it satisfies (1) too.
		
		First let's list some properties of $ s $. We have,
		\begin{enumerate}
			\itemsep0em
			\item $ s(I_2)=I_2 $ where $ I_2 $ is the identity matrix.
			\item $ s(XY) = s(Y)s(X) $, trivial by \autoref{lemma:Multiplication Order}
			\item $ s(A)=A \text{ and } s(B)=B $, easy to show with some computation.
			\item It follows that $ s(K_i)=K_i $.
		\end{enumerate}
		(1) trivially follows from these properties. So $ s $ satisfies the problem.
	\end{solution}
	
	
	\rem{The above solution followed a rather standard procedure (translating linear recurrences into matrix multiplication - this is the same trick that solves many problems about Fibonacci numbers) until the point where we "guessed" the matrix $U$ and the map $s$. How did we do that?
		
		The motivation is the following: We need a map $s$ which satisfies (1) and (2). We forget about (2) for a moment, and try to satisfy (1) only.
		
		The easiest way to ensure that (1) holds for every choice of $n$ and $\varepsilon_1,\varepsilon_2,...,\varepsilon_{n-1}$ is to choose $s$ as a linear map satisfying $s\left(A\right)=A$ and $s\left(B\right)=B$ (this immediately guarantees that $s\left(K_i\right) = K_i$ for every $i$, because $K_i=\varepsilon_i B + \left(1-\varepsilon_i\right) A $ is a linear combination of $A$ and $B$) and satisfying $s\left(X_kX_{k-1}...X_1\right) = s\left(X_1\right) s\left(X_2\right) ... s\left(X_k\right)$ for any $k\in \mathbb N$ and any $2\times 2$ matrices $X_1$, $X_2$, ..., $X_k$. 
		
		This condition $s\left(X_kX_{k-1}...X_1\right) = s\left(X_1\right) s\left(X_2\right) ... s\left(X_k\right)$ is fulfilled, for example, when the map $s$ has the form $s\left(P\right) = UP^TU^{-1}$ for every $P\in\mathrm{M}_2\left(\mathbb Q\right)$ for $U$ an invertible $2\times 2$ matrix. Actually it is fulfilled only in this case, as I explain further below, but as for now let us at least agree that $s\left(P\right) = UP^TU^{-1}$ for every $P\in\mathrm{M}_2\left(\mathbb Q\right)$ is a good point to start.
		
		So now we are searching for a $2\times 2$ matrix $U$ such that the map $s$ defined by $s\left(P\right) = UP^TU^{-1}$ for every $P\in\mathrm{M}_2\left(\mathbb Q\right)$ satisfies $s\left(A\right)=A$, $s\left(B\right)=B$ and (2). These conditions give linear equations on the entries of this matrix $U$, and the only matrix $U$ which solves all of them is (up to scaling) $\left(\begin{array}{cc} 7&1\\ 1&2\end{array}\right) \in \mathrm{M}_2\left(\mathbb Q\right)$. It is now clear how to proceed from here.}
	
	
	\begin{solution}[Bijection, evan chen]
		Let $ A_i=2^ia_i $. So the recursive relations now are:
			\[A_{i+1}=8A_{i-1}+6A_i \text{ or } 12A_{i-1}+2A_i\]
		Now we build a bijective relation. Imagine $ n $ rooms in a row with $ n-1 $ doors, numbered $ \epsilon_i $, between them. We have $ 14 $ colors, $ \{\star, 1, 2, \dots 13\} $ to paint the rooms. So the two sides of the doors will get different coloring, let these two colors be $ i, j $. We need to follow some certain rules:
			\begin{enumerate}
				\itemsep-.4em
				\item If $ i $ is $ \star $, then $ j $ can be any color.
				\item If the door is labeled with $ 0 $, and $ i \in \{\star, j-2, j-1, j, j+1, j+2\} \pmod {13} $
				\item If the door is labeled with $ 1 $, $ i \in \{\star, j\} $.
			\end{enumerate}
		We prove that $ A_i $ is the number of ways the first $ i $ rooms can be painted (and $ B_i $ the number of ways the last $ i $ rooms).
		
		First, the door between the rooms $ i, i+1 $ was labeled $ 0 $. If $ A_i\ne \star $, then we have $ 6 $ ways to paint $ i+1 $. From there comes $ 6A_i $ ways of painting the first $ i+1 $ rooms. If $ A_i=\star $, then there are $ 14 $ ways of painting $ i+1 $. But $ 6 $ of those ways have already been counted. And the rest of the $ 8 $ new colors will come from $ A_{i-1} $. So $ 8A_{i-1} $ ways from there. In total:
			\[A_{i+1} = 8A_{i-1}+6A_i\]
		And if the door was labeled $ 1 $, then $ 6, 8 $ becomes $ 2, 12 $. And our result follows.
	\end{solution}

	\rem{We first have to settle to search for a bijective solution. So we ask the question, ``$ a_i $'s are the number of ways to do something. How can we define that?''
		
	Now the coefficients of the recurresions doesn't add up. It would be better if we had them being equal. So we do that, nice and easy. 
	
	So the coefficients now add up to $ 14 $. So for the $ i+1 $-th object, we have $ 14 $ choices, and those choices get divided depending on the choice for the $ i $-th object. From here, it is natural to think about those rules.}
	
	
	
	

\newpage\section{Double Counting and Other Algebraic Methods}

	\prob{https://artofproblemsolving.com/community/c6h546170p3160563}{ISL 2012 C3}{M}{In a $999 \times 999$ square table some cells are white and the remaining ones are red. Let $T$ be the number of triples $(C_1,C_2,C_3)$ of cells, the first two in the same row and the last two in the same column, with $C_1,C_3$ white and $C_2$ red. Find the maximum value $T$ can attain.}
		
		\solu{Expicitly count the value, and bound it using ineq...}




\newpage\subsection{Probabilistic Methods}

	\prob{https://artofproblemsolving.com/community/c6h155694p874985}{ISL 2006 C3}{M/E}{Let $ S$ be a finite set of points in the plane such that no three of them are on a line. For each convex polygon $ P$ whose vertices are in $ S$, let $ a(P)$ be the number of vertices of $ P$, and let $ b(P)$ be the number of points of $ S$ which are outside $ P$. A line segment, a point, and the empty set are considered as convex polygons of $ 2$, $ 1$, and $ 0$ vertices respectively. Prove that for every real number $ x$ \[\sum_{P}{x^{a(P)}(1 - x)^{b(P)}} = 1,\] where the sum is taken over all convex polygons with vertices in $ S$.}
	
		\solu{This can be done by strong induction and double counting. Counting for every possibile subsets with every points inside of the convex hull and at least one point on the convex hull outside of the set.}
		
		\solu{The beautiful solution on the other hand uses probability. Color each point black or white, then translate the condition in terms of probability. :D}
		