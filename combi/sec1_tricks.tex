\newpage\section{Tricks}


	\Faka\Faka\subsection{Bijection}

	{

	Ideas for the bijection function:

	\begin{itemize}
 
		\item Induction
		\item Forming sets that are not already formed
		\item Building combinatorial models from the investigation of the problem conditions.
		\item Trying to define the later set by the former set.

	\end{itemize}


	}\faka



	\begin{enumerate}[wide=0em, label=\arabic*, itemsep=0pt, parsep=0pt, font=\footnotesize\bfseries]

		\iref{problem:bijection_1}{ISL 2002 C1,}{Red-Blue Under $ x+y < n $ and Bijection}
		\iref{problem:bijection_2}{OC Chap2 P2,}{Magic trick of hiding two digits}
		\iref{problem:bijection_3}{ISL 2009 C3}{}
		\iref{problem:bijection_4}{USAMO 1996 P4,}{Binary Strings NOT containing certain Combinations}
		\iref{problem:bijection_5}{ISL 2008 C4,}{Lamp States and Probability}
		\iref{problem:bijection_6}{APMO 2017 P3,}{Bijection Problem}
		\iref{problem:bijection_7}{USAMO 2013 P2,}{Around the circle on points with move or $ 1 $ or $ 2 $}
		\iref{problem:bijection_8}{APMO 2008 P2}{}
		\iref{problem:bijection_9}{ISL 2006 C2}{}
		\iref{problem:bijection_10}{USA TST 2009 P1}{}
		\iref{problem:bijection_11}{ISL 2005 C3}{}
		\iref{problem:bijection_12}{ISL 2002 C2}{Cover all black squares with L-tromino}
		\iref{problem:bijection_13}{ISL 2002 C3}{Full-Sequences}
	\end{enumerate}








	\Faka\subsubsection{Hall's Marriage Lemma\label{hall_marriage}}

	\begin{enumerate}[wide=0em, label=\arabic*, itemsep=0pt, parsep=0pt, font=\footnotesize\bfseries]

		\iref{problem:hall_marriage_1}{OC Chap2 P2}{}
		\iref{problem:hall_marriage_2}{ARO 2005 P9.4}{}
	\end{enumerate}





	\subsection{Extremal Principal}




	\Faka\subsubsection{Whole Extremal Cases\label{extremal_case_whole}}{Exploring the extreme case as a whole}

	\begin{multicols}{2}
		\begin{enumerate}[wide=0em, label=\arabic*, itemsep=0pt, parsep=0pt, font=\footnotesize\bfseries]

			\iref{problem:extremal_case_whole_1}{ISL 2013 C1}{}

			\iref{problem:extremal_case_whole_2}{Brazilian Olympic Revenge 2014}{}
			\iref{problem:extremal_case_whole_3}{ISL 2009 C1}{}
			\iref{problem:extremal_case_whole_4}{EGMO 2017 P2}{}
			\iref{problem:extremal_case_whole_5}{APMO 2008 P2}{}
			\iref{problem:extremal_case_whole_6}{Belarus 2001}{}
			\iref{problem:extremal_case_whole_7}{ISL 2014 C3}{}
			\iref{problem:extremal_case_whole_8}{USA TST 2017 P1}{}
			\iref{problem:extremal_case_whole_9}{Polynomials and Roots problem}{}
			\iref{problem:extremal_case_whole_10}{USAMO 2006 P2}{}
			\iref{problem:extremal_case_whole_11}{ISL 2014 N3,}{Cape Town Coin problem}
		\end{enumerate}
	\end{multicols}



	\Faka\subsubsection{Forget and Focus}{Explore only one part of the problem at a time, choose the most crucial part of the problem and focus only on that.}\label{forget_and_focus}

	\begin{multicols}{2}
		\begin{enumerate}[wide=0em, label=\arabic*, itemsep=0pt, parsep=0pt, font=\footnotesize\bfseries]

			\iref{problem:forget_and_focus_1}{ARO 2018 P9.5}{}
			\iref{problem:forget_and_focus_2}{Polish OI}{}
			\iref{problem:forget_and_focus_3}{ARO 2008 P9.5}{}
			\iref{problem:forget_and_focus_4}{ISL 2001 C6}{}
			\iref{problem:forget_and_focus_5}{Swell Coloring}{}
			\iref{problem:forget_and_focus_6}{Indian Postal Coaching 2011}{}
			\iref{problem:forget_and_focus_7}{Romanian TST 2016 D1P2,}{associating $ x_i $ with $ S_i $}

		\end{enumerate}
	\end{multicols}



		\Faka\paragraph{Swapping / Forget and Focus (2)}{Focusing on two neighboring elements in the extremal case.}\label{swapping}

		\begin{multicols}{2}
			\begin{enumerate}[wide=0em, label=\arabic*, itemsep=0pt, parsep=0pt, font=\footnotesize\bfseries]

				\iref{problem:swapping_1}{Polish OI}{}
				\iref{problem:swapping_2}{IOI 2007 P3}{}
				\iref{problem:swapping_3}{Problem}{}
				\iref{problem:swapping_4}{ARO 2014 P9.8}{}
				\iref{problem:swapping_5}{Problem,}{Robot goes at a speed of $ i $ for $ i $ seconds in mode $ i $}
				\iref{problem:swapping_6}{Problem,}{The same with speed $ n-i $}
			\end{enumerate}
		\end{multicols}


		\Faka\paragraph{Game Positions}{Considering Winning/Losing positions and describing the game with these definitions is an important game theory tactic.}\label{win_lose}

		\begin{multicols}{3}
			\begin{enumerate}[wide=0em, label=\arabic*, itemsep=0pt, parsep=0pt, font=\footnotesize\bfseries]

				\iref{problem:win_lose_1}{ISL 2004 C5}{}
			\end{enumerate}
		\end{multicols}


	


	\Faka\subsubsection{Extreme Objects}{Concentrating on the Extreme object only}\label{extreme_object}

	\begin{enumerate}[wide=0em, label=\arabic*, itemsep=0pt, parsep=0pt, font=\footnotesize\bfseries]

		\iref{problem:extreme_object_1}{USAMO 2008 P4}{}
		\iref{problem:extreme_object_2}{ISL 2008 C1}{}
		\iref{problem:extreme_object_3}{IMO 2011 P2}{}
		\iref{problem:extreme_object_4}{Iran TST 2002 P3}{}
		\iref{problem:extreme_object_5}{Problem}{}
		\iref{problem:extreme_object_6}{ISL 2006 C4}{}
		\iref{problem:extreme_object_7}{ISL 2014 C1}{}
		\iref{problem:extreme_object_8}{ARO 1993 P10.4}{}
		\iref{problem:extreme_object_10}{Sunflower Lemma}{}
		\iref{problem:extreme_object_11}{ISL 1998 C4}{}
		\iref{problem:extreme_object_12}{USA TST 2009 P1}{}
		\iref{problem:extreme_object_13}{AoPS}{}
		\iref{problem:extreme_object_14}{APMO 2012 P2}{}
		\iref{problem:extreme_object_15}{Problem,}{Confidence in solving a P6}
		\iref{problem:extreme_object_16}{ISL 2001 C3,}{$ 3 $-cliques with common points}
		\iref{problem:extreme_object_17}{ISL 2007 C2,}{dissecting a rectangle into $ n $ smaller rectangles, there exists a rectangle inside}


	\end{enumerate}




	\subsection{Coloring}


		\begin{itemize}

			\item If the nodes are connected in lattice point manner, then \textbf{Checkerboard} coloring is the most natural coloring technique. But if this coloring does not do any good, then there may be other alternatives and derivatives of checkerboard, like \textbf{Pseudo Checkerboard} or \textbf{Double Checkerboard}. The Pseudo Checkerboard's each row (or column) starts and end with the same color (If there are odd nodes in each row). In a Double Checkerboard, two consecutive nodes are of the same color. (You get the picture, don't you?)

			\item Checkerboard with $ \frac{1}{2}\times \frac{1}{2} $ sized cells. Proof of the rectangle with integer side problem.

			\item Color with ``Roots of Unity''.

			\item A knight's move always changes the color of the cell.

		\end{itemize}



	\begin{enumerate}[wide=0em, label=\arabic*, itemsep=0pt, parsep=0pt, font=\footnotesize\bfseries]

		\iref{problem:coloring_1}{USAMO 2014 P1}{}
		\iref{problem:coloring_2}{USAMO 2008 P3}{}
		\iref{problem:coloring_3}{IMO 2018 P4}{}
		\iref{problem:coloring_4}{Codeforces 101954/G}{}

	\end{enumerate}


	\Faka\subsubsection{Plane divided by lines}{In problems regarding the plane being divided by straight lines, color the plane with chessboard colors.}\label{plane_coloring}

	\begin{multicols}{3}
		\begin{enumerate}[wide=0em, label=\arabic*, itemsep=0pt, parsep=0pt, font=\footnotesize\bfseries]

			\iref{problem:plane_coloring_1}{EGMO 2017 P3}{{}}
		\end{enumerate}
	\end{multicols}





	\subsection{Divide and Conquer\label{divide_and_conquer}}

	\vspace{10mm}


	Divide the problem/grid/graph into smaller pieces and work through them separately and finally join them together. The main difference between this and induction/recursion is that we have to actually work in the smaller cases instead of assuming that they are true.

	\begin{enumerate}[wide=0em, label=\arabic*, itemsep=0pt, parsep=0pt, font=\footnotesize\bfseries]

		\iref{problem:divide_and_conquer_1}{USA TST 2011 P2,}{Capacity $ 1, 2 $ roads}
		\iref{problem:divide_and_conquer_2}{CodeForces 744B,}{Finding the minimum number in the rows}
		\iref{problem:divide_and_conquer_3}{Problem,}{Double binary search}
		\iref{problem:divide_and_conquer_4}{ISL 2005 C1,}{Lamps in rooms}
		\iref{problem:divide_and_conquer_5}{IOI 2016 P5,}{Bug changes the strings}
		\iref{problem:divide_and_conquer_6}{Iran TST 2007 P2,}{$x$ divides at most one other element in $A$}
	\end{enumerate}




	\Faka\subsubsection{Induction}{\textbf{Cauchy Induction}: $ n \rightarrow 2n, n \rightarrow n-1 $ }\label{induction}


		Can be used in almost any kind of problems, often called `\textit{goriber bondhu}'.

		\begin{itemize}

			\item In MO probs $ 2-3-5-6 $ or SL $ 3+ $ (often $ 1, 2 $ as well) you can be sure that applying only induction isn't going to do any good. You'll need extra tools, and you might need to apply induction more than once.

			\begin{itemize}

				\item Sometimes, in graph probs, apply indution on more than one node gives better results.
				\item Often you can set up your induction in more than one way, and finding the right way makes the problem much simpler.
				\item Sometimes trying to prove more by adding a stronger induction hypothesis makes it easier to carry out the induction.

			\end{itemize}

		\end{itemize}


	\faka\textbf{Type 1}: $ n-1 \rightarrow n $

	\begin{enumerate}[wide=0em, label=\arabic*, itemsep=0pt, parsep=0pt, font=\footnotesize\bfseries]

		\iref{problem:induction_type1_1}{ISL 2004 C2,}{$ n $ circles intersect, colors}
		\iref{problem:induction_type1_2}{ISL 1997 P4,}{Silver matrix}
		\iref{problem:induction_type1_3}{ARO 2018 P11.5,}{an easy graph}
		\iref{problem:induction_type1_4}{ISL 2006 C1,}{lamps will eventually be off}
		\iref{problem:induction_type1_5}{ISL 2002 C1,}{bijection in $ x+y=n $ and red-blue colors}
		\iref{problem:induction_type1_6}{ISL 2012 C2}{}
		\iref{problem:induction_type1_7}{ARO 2013 P9.4}{}
		\iref{problem:induction_type1_8}{ISL 2005 C2}{}
		\iref{problem:induction_type1_9}{ISL 2013 C3}{}
		\iref{problem:induction_type1_10}{ISL 2005 C1}{}
		\iref{problem:induction_type1_11}{ISL 2016 C6,}{the ferry problem}
		\iref{problem:induction_type1_12}{IMO SL 1985}{}
		\iref{problem:induction_type1_13}{ELMO 2017 P5}{}
		\iref{problem:induction_type1_14}{ISL 1990}{}
		\iref{problem:induction_type1_15}{USAMO 2017 P4}{}
		\iref{problem:induction_type1_16}{Jacob Tsimerman Induction}{}
		\iref{problem:induction_type1_17}{All Russia 2017 9.1}{}
		\iref{problem:induction_type1_18}{Iran TST 2008 D3P1}{}
		\iref{problem:induction_type1_19}{USA TST 2011 D3P2}{}
		\iref{problem:induction_type1_20}{Sunflower Lemma}{}
		\iref{problem:induction_type1_21}{ISL 1998 C4}{}
		\iref{problem:induction_type1_22}{Generalization of USAMO 1999 P1}{}
		\iref{problem:induction_type1_23}{USAMO 2005 P1,}{arranging divisors on a circle with no co-prime neighbors}
		\iref{problem:induction_type1_24}{USAMO 2006 P5,}{a frog jumps jumps of $ 2 $-powers}
		\iref{problem:induction_type1_25}{Romanian TST 2016 D1P2,}{associating $ x_i $ with $ S_i $}
		\iref{problem:induction_type1_26}{American Mathematical Monthly,}{$ n $ subsets from $ S={1\dots n-1} $ and a weird relation}
		\iref{problem:induction_type1_27}{ISL 1991 P10,}{Color the graph by numbers such that any vertex is gcd $ 1 $}
		\iref{problem:induction_type1_28}{Problem,}{Circles $ 1 $ unit apart, gotta cover them up.}
		\iref{problem:induction_type1_29}{IOI 2018 P1,}{Prefixes of a string}
		\iref{problem:induction_type1_30}{German TST 2004 E7P3,}{A white graph to a black graph}
		\iref{problem:induction_type1_31}{ISL 2002 C5,}{An finite family of sets of size $ r $ has a intersecting set of size $ r-1 $}
		\iref{problem:induction_type1_32}{US Dec TST 2016, P1,}{$ k $ bijections, and cycles in those}
	\end{enumerate}



	\faka\textbf{Type 2}: $ k (k<n-1) \rightarrow n $

	\begin{enumerate}[wide=0em, label=\arabic*, itemsep=0pt, parsep=0pt, font=\footnotesize\bfseries]

		\iref{problem:induction_type2_1}{ARO 2014 P9.3}{}
		\iref{problem:induction_type2_2}{Mexican Regional 2014 P6}{}
		\iref{problem:induction_type2_3}{USA TST 2011 P2}{}
		\iref{problem:induction_type2_4}{ISL 2006 C2}{}
		\iref{problem:induction_type2_5}{APMO 1999 P2}{$a_{i+j} \leq a_i+a_j$}
	\end{enumerate}




	\Faka\subsubsection{Inductive/Recursive Relations}{Building other solutions depending on already or easily tweakable solutions.}\label{recursive_solution}


		\begin{enumerate}[wide=0em, label=\arabic*, itemsep=0pt, parsep=0pt, font=\footnotesize\bfseries]

			\iref{problem:recursive_solution_1}{India TST 2013 Test 3, P1}{}
			\iref{problem:recursive_solution_2}{ISL 2002 C1}{}
			\iref{problem:recursive_solution_3}{ISL 2009 C3}{}
			\iref{problem:recursive_solution_4}{USAMO 2013 P2}{}
			\iref{problem:recursive_solution_5}{IMO 2011 P4}{}
			\iref{problem:recursive_solution_6}{ARO 2014 P9.7,}{stable coin system with coins of value $ \a^k $}
			\iref{problem:recursive_solution_7}{Saint Petersburg 2001}{}
		\end{enumerate}





		\Faka\paragraph{Catalan Numbers}{$ C_n = \frac{1}{n+1}\binom{2n}{n} $, this little number is associated with a lot of combinatorial setups. And has the \hrf{lemma:catalan_recursion}{recursion}.}\label{catalan_numbers}




	\subsection{Count the Shit Up}


	\Faka\subsubsection{Double Counting\label{double_counting}}

	Explicitly count the number of things, but Twice!

	\begin{enumerate}[wide=0em, label=\arabic*, itemsep=0pt, parsep=0pt, font=\footnotesize\bfseries]

		\iref{problem:double_counting_1}{ISL 2016 C3,}{$ n $-gon colored with $ 3 $ colors, exists isosceles triangle.}
		\iref{problem:double_counting_2}{Iran TST 2012 P4,}{Path inside of a $ m\times n $.}
		\iref{problem:double_counting_3}{ISL 2014 C1,}{Dissecting a Rectangle wrt to some given points inside.}
		\iref{problem:double_counting_4}{Problem,}{$ 10 $ person bookstore problem.}
		\iref{problem:double_counting_5}{USA TST 2005 P1,}{Subsets of the set $ \{1, 2,\dots, mn\} $}
		\iref{problem:double_counting_6}{USAMO 2012 P2,}{$ 4 $ colors on the circle, rotating.}
		\iref{problem:double_counting_7}{ISL 2004 C1,}{A Cauchy type function based on a coloring of the integers.}
		\iref{problem:double_counting_8}{China MO 2018 P2,}{$ 3n^2-3n+1+k $ points are red, $ k $ good boxes exist.}
		\iref{problem:double_counting_9}{Problem}{}
		\iref{problem:double_counting_10}{ISL 2003 C3}{}

	\end{enumerate}





	\faka\subsubsection{Generating Function\label{generating_function}}

	For a sequence $ A = (a_0, a_1, a_2 \dots) $, the generating function$ E_A(x) $ for this sequence is of several types types:

	\begin{enumerate}[wide=0em, label=\arabic*, itemsep=0pt, parsep=0pt, font=\footnotesize\bfseries]

		\item $ E_A(x) = a_0x^0 + a_1x^1 \dots + a_ix^i \dots $, Useful for usual recursive sequences.

		\item $ E_A(x) = x^{a_0} + x^{a_1} \dots + x^{a_i} \dots $, Useful for sum of any two elements from \emph{any} two sequences.

		\item $ E_A(x) = \prod (1 + x^{a_i}) $, Useful for sum of multiple numbers from one sequence.

		\item $ E_A(t, x) = \prod (t + x^{a_i}) $, Useful for keeping track of how many numbers are being added to the sum.

	\end{enumerate}



	\begin{enumerate}[wide=0em, label=\arabic*, itemsep=0pt, parsep=0pt, font=\footnotesize\bfseries]

		\iref{problem:generating_function_1}{Problem,}{two sequence's pairwise sum's tuples are the same, $ n=2^k $.}
		\iref{problem:generating_function_2}{Result by Erdos,}{Partitioning the integers into arithmetic sequences.}
		\iref{problem:generating_function_3}{Problem on Catalan's Recursion}{}
	\end{enumerate}






		\faka\paragraph{Roots of Unity}{These things can be used in a lot of places, like coloring boards, coloring (dividing into modular classes) the integers, using as variables in generating functions etc.}\label{roots_of_unity}


		\begin{multicols}{2}
			\begin{enumerate}[wide=0em, label=\arabic*, itemsep=0pt, parsep=0pt, font=\footnotesize\bfseries]

				\iref{problem:roots_of_unity_1}{Result by Erdos}{}
			\end{enumerate}
		\end{multicols}





	\subsection{Different Representation}

	Represent the problem or the problem objects differently, usually by binary strings, graphs or matrices.




		\Faka\subsubsection{Binary}\label{binary}

		Associate a binary string to elements, like for handling subsets, add a string to each element, representing if it is in a certain set or not.

			\begin{enumerate}[wide=0em, label=\arabic*, itemsep=0pt, parsep=0pt, font=\footnotesize\bfseries]

				\iref{problem:binary_1}{IOI Practice 2017}{}
				\iref{problem:binary_2}{ISL 1988 P10}{}
			\end{enumerate}




		\Faka\subsubsection{Binary Query}\label{binary_query}

		Asking questions of the kind: if in the binary expansion of $ k $ , if the $ i $ th bit is $ 0 $ , add $ k $ to one kind of query and if it is $ 1 $ , then add it to another.

			\begin{enumerate}[wide=0em, label=\arabic*, itemsep=0pt, parsep=0pt, font=\footnotesize\bfseries]

				\iref{problem:binary_query_1}{CodeForces 744B}{}
				\iref{problem:binary_query_2}{Problem}{}
			\end{enumerate}




		\Faka\subsubsection{Matrix Creation}\label{matrix_creation}

		When there is some sort of $ a\times b $ always try to create a matrix.

			\begin{enumerate}[wide=0em, label=\arabic*, itemsep=0pt, parsep=0pt, font=\footnotesize\bfseries]

				\iref{problem:matrix_creation_1}{IMO 2017 P5}{}
			\end{enumerate}




		\Faka\subsubsection{Graph}\label{graph_representation}


		Problems concerning sets and their relations, consider representing using graph, with some fixed mapping rules. Some times changing grid cells into vertices's also helps.


			\begin{enumerate}[wide=0em, label=\arabic*, itemsep=0pt, parsep=0pt, font=\footnotesize\bfseries]

				\iref{problem:graph_representation_1}{USAMO 1986 P2}{}
				\iref{problem:graph_representation_2}{Mexican Regional 2014 P6}{}
				\iref{problem:graph_representation_3}{AoPS}{}
				\iref{problem:graph_representation_4}{ARO 2013 P9.5}{}
				\iref{problem:graph_representation_5}{ISL 2002 C6}{}
				\iref{problem:graph_representation_6}{Problem,}{Mailman messes up}
				\iref{problem:graph_representation_7}{timus 1862,}{Sum of operations}
				\iref{problem:graph_representation_8}{ARO 2007 P9.7,}{Adding diagrams that cut even number of already drawn ones}
				\iref{problem:graph_representation_9}{ISL 2002 C3}{Full-Sequences}
			\end{enumerate}


		\Faka\subsubsection{Changing the Target Term}

		Changing the term you have to achieve to a slightly more intuitive one, usually thinking about what values you can get more naturally from the given conditions, and to build a similar term from the original term.

			\begin{enumerate}[wide=0em, label=\arabic*, itemsep=0pt, parsep=0pt, font=\footnotesize\bfseries]

				\iref{problem:changing_term_1}{ARO 2014 P10.8,}{Mutually intersecting $ k $-gons, one point inside of a bunch of gons}{}

			\end{enumerate}








	\subsection{Algorithms}


		\Faka\subsubsection{Greedy Algorithm}\label{greedy_algorithm}


		\begin{enumerate}[wide=0em, label=\arabic*, itemsep=0pt, parsep=0pt, font=\footnotesize\bfseries]

			\iref{problem:greedy_algorithm_1}{ISL 2014 N3,}{Cape Town Coin problem}
			\iref{problem:greedy_algorithm_2}{China TST 2006}{}
			\iref{problem:greedy_algorithm_3}{Timus 1578}{}

		\end{enumerate}





		\Faka\subsubsection{Constructive Algorithm}\label{constructive_algo}


		In these kinda problems, you have to prove using a construction. In other words, proof by \emph{Existence}. The key is to add one object to the solution set one at a time depending on already added objects in the set and maintaining the problem conditions. Sometimes by adding additional constraints or prioritizing already given constraints.


		\begin{enumerate}[wide=0em, label=\arabic*, itemsep=0pt, parsep=0pt, font=\footnotesize\bfseries]

			\iref{problem:constructive_algo_1}{Bulgarian IMO TST 2004, D3P3}{}
			\iref{problem:constructive_algo_2}{ARO 2018 10.3}{}
			\iref{problem:constructive_algo_3}{CodeForces 960/C}{}
			\iref{problem:constructive_algo_4}{ARO 2005 P10.3, P11.2}{}
			\iref{problem:constructive_algo_5}{IOI 2007 P3}{}
			\iref{problem:constructive_algo_6}{Problem}{}
			\iref{problem:constructive_algo_7}{ARO 2018 P11.5}{}
			\iref{problem:constructive_algo_8}{ARO 2013 P9.4}{}
			\iref{problem:constructive_algo_9}{ARO 2014 P9.8}{}
			\iref{problem:constructive_algo_10}{ISL 2016 C1}{}
			\iref{problem:constructive_algo_11}{India IMO Camp 2017}{}
			\iref{problem:constructive_algo_12}{ISL 2012 C2}{}
			\iref{problem:constructive_algo_13}{CodeForces 989C}{}
			\iref{problem:constructive_algo_14}{CodeForces 989B}{}
			\iref{problem:constructive_algo_15}{ISL 2011 A5}{}
			\iref{problem:constructive_algo_16}{Iran TST 2017 D1P1}{}
			\iref{problem:constructive_algo_17}{ISL 2009 C2}{}
			\iref{problem:constructive_algo_18}{Problem}{}
			\iref{problem:constructive_algo_19}{Putnam 2017 A4}{}
			\iref{problem:constructive_algo_20}{Serbia TST 2017 P4}{}
			\iref{problem:constructive_algo_21}{ISL 2014 A1}{}
			\iref{problem:constructive_algo_22}{ISL 2005 N2,}{sequence that contains all of the integers}
			\iref{problem:constructive_algo_23}{Problem,}{switch states of a row and column}
			\iref{problem:constructive_algo_24}{USAMO 2015 P4,}{piles of stone on cells, mone on the corners of a rectangle}
			\iref{problem:constructive_algo_25}{ISL 2003 C4}{}

		\end{enumerate}



		\Faka\subsubsection{Element : Time}\label{add_time}

		Adding an element of Time to give the static problem a dynamic view. In less formal words, if a problem environments seems to just *exist*, add a dynamic way to slowly visualize the environment to exist. One kind of constructive algorithm, but this algo doesn't build the answer or solution, instead it builds up the whole environment step by step.


		\begin{enumerate}[wide=0em, label=\arabic*, itemsep=0pt, parsep=0pt, font=\footnotesize\bfseries]

			\iref{problem:add_time_1}{ARO 2016 P3}{}
			\iref{problem:add_time_2}{Brazilian Olympic Revenge 2014}{}
			\iref{problem:add_time_3}{ISL 2008 C1}{}
			\iref{problem:add_time_4}{USAMO 1999 P1}{}
			\iref{problem:add_time_5}{AoPS}{}

		\end{enumerate}



		\subsubsection{Gaming Tricks}


			\paragraph{Pairing and Copying}

			{Who said you can't cheat in a combinatorial game? Just follow your opponents movements, and copy them cleverly.}


			\begin{enumerate}[wide=0em, label=\arabic*, itemsep=0pt, parsep=0pt, font=\footnotesize\bfseries]

				\iref{problem:pairing_and_copying_1}{ARO 2011 P11.6,}{Take a number of stones off the heap of size $ n^2 $}

			\end{enumerate}


			\paragraph{Nim Equivalence}





	\subsection{Invariance Rules of Thumb}\label{invariant_rules_of_thumb}

		\begin{enumerate}[wide=0em, label=\arabic*, itemsep=0pt, parsep=0pt, font=\footnotesize\bfseries]

			\item Natural Sum
			\item Alternating Sum
			\item Sum of Squares
			\item Product

			\item Giving weight to each of the elements, in problems where usually no trivial invariants exists. Like weights of $ 2^i, \frac{1}{i}, \text{roots of unity} $ etc. depending on the problem's nature.
		\end{enumerate}

	\begin{itemize}

		\iref{problem:invariant_rules_of_thumb_1}{ISL 2014 C2,}{$ 1 $ written on $ 2^m $ papers and and an addition operation}
		\iref{problem:invariant_rules_of_thumb_2}{ISL 2012 C1,}{Operation almost alike to swapping and sorting}
		\iref{problem:invariant_rules_of_thumb_3}{Indian TST 2004,}{Pebble makes a clone and moves up and right}
		\iref{problem:invariant_rules_of_thumb_4}{ISL 1998 C7,}{One lamp on each cell, switching one lamp switches neighbors}
		\iref{problem:invariant_rules_of_thumb_5}{APMO 2017 P1,}{$ a-b+c-d+e=29 $}
		\iref{problem:invariant_rules_of_thumb_6}{AoPS}{}
		\iref{problem:invariant_rules_of_thumb_7}{ARO 2016 P1}{}
		\iref{problem:invariant_rules_of_thumb_8}{Serbia TST 2017 P2,}{$ (x+y) \rightarrow (\frac{x}{2}, y+\frac{x}{2}) $ or $ (x+\frac{y}{2}, \frac{y}{2}) $}
		\iref{problem:invariant_rules_of_thumb_9}{ISL 1994 C3,}{$ 3 $ bank accounts}
		\iref{problem:invariant_rules_of_thumb_10}{Codeforces 987E}{}
		\iref{problem:invariant_rules_of_thumb_11}{ISL 2007 C4,}{Dividing a sequence with almost equal sum, and getting another sequence from it.}
		\iref{problem:invariant_rules_of_thumb_12}{USAMO 2015 P4,}{piles of stone on cells, mone on the corners of a rectangle}

	\end{itemize}



	\Faka\subsubsection{Monotonicity with strict constraints\label{monotonicity_with_constraints}}

	If regular monotonicity doesn't apply, then some monotonicity with special properties might work. Like keeping the sum \emph{even}, \emph{odd}, \emph{square} etc.


	\begin{enumerate}[wide=0em, label=\arabic*, itemsep=0pt, parsep=0pt, font=\footnotesize\bfseries]

		\iref{problem:monotonicity_with_constraints_1}{USAMO 2013 P6,}{Replace $ x $ with the difference of the two neighboring numbers.}
		\iref{problem:monotonicity_with_constraints_2}{MEMO 2008, Team, P6,}{$ a, b \rightarrow a+b, a+b $}

	\end{enumerate}








	\subsection{Pigeonhole Principal}\label{php}



	\Faka\subsubsection{Alternating Chains Technique}{In a Cyclic graph with $ n $ nodes, if your task is to color some of the nodes so that no two neighboring nodes will be colored, you can color at most $ \floor{\frac{n}{2}} $. And in a path, this value is $ \ceil{\frac{n}{2}} $ }\label{alternating_chains}

	\begin{multicols}{3}
		\begin{enumerate}[wide=0em, label=\arabic*, itemsep=0pt, parsep=0pt, font=\footnotesize\bfseries]

			\iref{problem:alternating_chains_1}{ISL 1990 P3}{}
			\iref{problem:alternating_chains_2}{USAMO 2008 P3}{}
		\end{enumerate}
	\end{multicols}











	\subsection{Other Useful Techniques and Philosophies}


	\Faka\subsubsection{Include potentially important players in the game}{If the problem condition is completely or partially but crucially depended on some problem object, but the proof condition doesn't directly depend on that object, think of a way to include that object in the proof condition.}\label{add_stuffs}


	\begin{multicols}{3}
		\begin{enumerate}[wide=0em, label=\arabic*, itemsep=0pt, parsep=0pt, font=\footnotesize\bfseries]

			\iref{problem:add_stuffs_1}{APMO 2017 P3}{}
			\iref{problem:add_stuffs_2}{USAMO 2008 P5}{}
		\end{enumerate}
	\end{multicols}



	\Faka\subsubsection{Finding the Tough Nut}{Solving an easier version of the problem with some sort of constraints lose, to find out exactly what makes the problem so tough. This way we get valuable information about on what our main focus should be.}\label{finding_the_tough_nut}








	\Faka\subsubsection{eChen trick}{Subtract a constant from all the numbers to make the sum $ 0 $. This makes the numbers easier to handle.}\label{minus_constant}

	\begin{multicols}{3}
		\begin{enumerate}[wide=0em, label=\arabic*, itemsep=0pt, parsep=0pt, font=\footnotesize\bfseries]

			\iref{problem:minus_constant_1}{APMO 2017 P1}{}
			\iref{problem:minus_constant_2}{ARO 2013 P9.5}{}
		\end{enumerate}
	\end{multicols}








	\Faka\subsubsection{Send objects to the infinity}{If there are too many arbitrary objects in the problem, try making some of them vanish by sending them to the infinity.}








	\Faka\subsubsection{n+1 = (n-i) + (i+1)}{ $ n+1 = (n-i) + (i+1) $ might prove to be useful when applying induction to $ \binom{n}{k} $ }






	\Faka\subsubsection{Convex Hulls, and Sandwiching two points}{You know what convex hull is. AND, One can draw two lines to separate two points - draw two parallel lines very close to the two points.}\label{sandwiching_points}


	\begin{enumerate}[wide=0em, label=\arabic*, itemsep=0pt, parsep=0pt, font=\footnotesize\bfseries]

		\iref{problem:sandwiching_points_1}{ISL 2013 C2}{}
		\iref{problem:convex_hull_1}{Putnam 1979,}{$ n $ red $ n $ blue, pair them}
		\iref{problem:convex_hull_3}{USAMO 2005 P5,}{$ n $ red $ n $ blue, at least two segments dividing them}
		\iref{problem:convex_hull_2}{ILL 1985}{}
	\end{enumerate}





