\newpage\section{Binomial Identities}



\theo{https://en.wikipedia.org/wiki/Vandermonde's_identity}{Vandermonde's identity}{
    For every positive integer $m, n, k$
    \begin{center}
        \tcboxmath[arc=0mm, colback=black!5!white, boxrule=0mm]{
            \binom{m+n}{k} = \sum_{i=0}^{k} \binom{m}{i}\binom{n}{k-i}
        }
    \end{center}
    \begin{center}
        \tcboxmath[arc=0mm, colback=black!5!white, boxrule=0mm]{
            {2n \choose n} = {n \choose 0}^2+ {n \choose 1}^2 \dots + {n \choose n}^2
        }
    \end{center}
}


\prob{https://artofproblemsolving.com/community/c6h1061914p4600670}{Somewhere}{M}{
    Let $n$ and $m$ be positive integers with $n<m$. Prove that 
    \begin{center}
        \tcboxmath[arc=0mm, colback=black!5!white, boxrule=0mm]{
        \sum_{k=0}^n (-1)^{n-k}\frac{1}{m-k}\binom{n}{k}=\frac{1}{(n+1)\binom{m}{n+1}}
        }
    \end{center} 
}

\solu{[MellowMelon]
    The left side is the $n$th finite difference of the unique $n$th degree
    polynomial satisfying $P(x) = \frac{1}{m-x}$ for $x = 0,1, \ldots, n$.
    This means that if $a_n$ is the $x^n$ coefficient of $P$, then the left
    side is equal to $n! \cdot a_n$. Setting this equal to the right side, it
    suffices to show that
    \[ a_n = \frac{1}{(n+1)! \binom{m}{n+1}} = \frac{1}{m(m-1)\cdots (m-n)}. \]

    Let $Q(x) = (x-m)P(x) + 1$. Then $Q$ is degree $n+1$, has roots
    $0,1,\ldots,n$, and has leading coefficient $a_n$. So
    \[ Q(x) = a_nx(x-1)(x-2)\cdots (x-n). \]
    We also know $Q(m) = 1$. Plugging $x = m$ into the above equation and
    isolating $a_n$ gives us exactly what we need to show.
}

\rem{
    The idea to express the left hand side with finite differences is pretty
    simple. The crux idea is to finding a $Q$ that works. Since we have to
    show $a_n\left(m\right)\left(m-1\right) \dots \left(m-n\right) =1$, which
    looks like a polynomial with leading coefficient $a_n$ with roots $0, 1, 2,
    \dots n$ and $Q(m) = 1$. How can we get that?

    The takeaway is, if the problem requires you to find a polynomial that
    almost looks like the one given in the problem, try to modify the given
    polynomial in some way first.

    In this particular case was to modify $P$ bu using the fact $(x-m)P(x) =
    -1$.
}

\solu{[Hydroxide]
    Let $P$ be the unique polynomial of degree at most $n$ such that
    $P(0)=P(1)=\cdots=P(n)=1$. Obviously $P(x)=1$, but we can also find $P$
    from Lagrange interpolation. This gives us the equality of polynomials

    \[ \sum_{k=0}^n \frac{(x-0)\cdots (x-(k-1))(x-(k+1))\cdots
    (x-n)}{(k-0)\cdots (k-(k-1))(k-(k+1))\cdots (k-n)}=1. \]

    Plugging in $x=m$ and doing a ton of rearranging gives the desired result.
}

\rem{
    Both solutions can easily be extended to prove the more general result
    that if $0\le p<n<m$ are integers, then
    \begin{center}
        \tcboxmath[arc=0mm, colback=black!5!white, boxrule=0mm]{
    \sum_{k=0}^n (-1)^{n-k}\frac{1}{m-k} \binom{n}{k}k^p=\frac{m^p}{(n+1)\binom{m}{n+1}} 
        }
    \end{center}
}

\prob{https://artofproblemsolving.com/community/c6h358774}{USA TST 2010
    P8}{M, 10/10}{
    Let $m,n$ be positive integers with $m \geq n$, and let $S$ be the
    set of all $n$-term sequences of positive integers $(a_1, a_2, \ldots
    a_n)$ such that $a_1 + a_2 + \cdots + a_n = m$. Show that 
    \begin{center}
        \tcboxmath[arc=0mm, colback=black!5!white, boxrule=0mm]{
            \sum_S 1^{a_1} 2^{a_2} \cdots n^{a_n} = {n \choose n} n^m - 
            {n \choose n-1} (n-1)^m + \cdots + (-1)^{n-2} {n \choose 2} 2^m +
            (-1)^{n-1} {n \choose 1}
        } 
    \end{center}
}
 

\solu{[Combinatorial, MellowMelon] 
    Look at the right side, try to translate it to
    combinatorial model. If we had used the inclusion-exclusion method, the
    right side would be the number of ways to color $ m $ balls with
    $ n $ colors, with each color appearing at least once. Now our remaining
    job is to prove the same for the left side.\\

    Trying to interpret the left hand side, we first notice that we need to
    partition the balls in $n$ parts with sizes $a_1, a_2, \dots  a_n$ where
    the first partition will have $1$ color to choose from, the second one
    will have $2$ and so on. We need to refine this idea for using it.
}


\solu{[Generating Function, MellowMelon]
    The first step of a gf solution is to decide on a gf solution. Now we can
    think of the left hand side as a function of $m$, namely $b_m$. So we can
    make a generating function with it.\\

    We let $F_n(x) = b_0 + b_1 x + b_2 x^2 + \cdots$ and so we get:
    \begin{align*}
        F_n(x) &= (x+x^2+\cdots)(2x+4x^2+\cdots)\cdots (nx+n^2x^2+\cdots), \\
               &= \frac{x}{1-x} \frac{2x}{1-2x} \cdots \frac{nx}{1-nx}, \\
               &= \frac{n! x^n}{(1-x)(1-2x) \cdots (1-nx)}. \\
    \end{align*}

    Now we need a generating function for the right hand side as well. The
    right hand side looks the $n$th finite difference of the function
    $_mg_0(x) = x^m$, that is 
    \[_mg_n(0) = {n \choose n} n^m - {n \choose n-1} (n-1)^m + \cdots +
    (-1)^{n-2} {n \choose 2} 2^m + (-1)^{n-1} {n \choose1}\] 

    Now we can make a generating function with this:
    \[G(y) =\ _0g_n(0)y^0 +\ _1g_n(0)y^1 \dots \]  
    But this isn't very useful as we are getting the $n$th finite differences
    on the right side, where we want to work with the original polynomials.\\

    So what we do is, we unravel the $_mg_n(0)$ as original polynomials and
    get:
    \[G(x, y) = x^0y^0 + x^1y^1 + x^2+y^2\dots \]
    Where our desired generating function for $n$ is:
    \[G_n(0, y) =\ _0g_n(0)y^0 +\ _1g_n(0)y^1 +\dots \] 
    
    And we want to show that 
    \[G_n(0, y) = \frac{n! x^n}{(1-x)(1-2x) \cdots (1-nx)} \] 

    Now we notice that we can actually use induction on $n$ since we definded
    \[G_{n+1}(x, y) = G_n(x+1, y) - G_n(x, y)\]

    And so we are done.
}


\lem{}{
    \[G(x)=\sum_{n=0}^{\infty} \binom{n+k}{k} x^n=\frac{1}{(1-x)^{k+1}}\]
}

\begin{prooof}
    We have the above relation because ${n+k \choose k}$ can be interpret as
    the number of ways write $n$ as the sum of $k+1$ numbers. And the
    generating function for that is:
    \[G(x) = (1+x+x^2\dots )^{k+1}\] 
\end{prooof}

\prob{https://artofproblemsolving.com/community/c6h1076934p5964577}{Binom1}{}{
    Solve for positive integer $n$
    \begin{center}
        \tcboxmath[arc=0mm, colback=black!5!white, boxrule=0mm]{
            {2n \choose 0}-{2n-1 \choose 1}+{2n-2 \choose 2}- \dots  +(-1)^n
            {n \choose n}=?
        }
    \end{center}
}

\solu{[Generating funtion, Tintarn]
    Denote this sum $a_n$. Denote $F(x)=\sum_{n=0}^{\infty} a_nx^{2n}$.
    \begin{align*}
        F(x)&=\sum_{k=0}^{\infty} (-1)^k \sum_{n=0}^{\infty} \binom{2n-k}{k}
        x^{2n}\\
        &=\sum_{k=0}^{\infty} (-x^2)^k \sum_{n=0}^{\infty} \binom{2n+k}{k}
        x^{2n}
    .\end{align*}
    Now, using the well-known power series
    \[\boxed{G(x)=\sum_{n=0}^{\infty} \binom{n+k}{k}
    x^n=\frac{1}{(1-x)^{k+1}}}\]
    we find
    \[\sum_{n=0}^{\infty} \binom{2n+k}{k} x^{2n}=\frac{G(x)+G(-x)}{2}\]
    and hence
    \begin{align*}
        F(x)&=\frac{1}{2(1-x)} \sum_{k=0}^{\infty} \left( \frac{x^2}{x-1} \right)^k
        +\frac{1}{2(1+x)} \sum_{k=0}^{\infty} \left( \frac{-x^2}{x+1} \right)^k\\[.5em]
            &=\frac{1}{2(x^2-x+1)}+\frac{1}{2(x^2+x+1)}\\[.5em]
            &=\frac{x^2+1}{x^4+x^2+1}\\[.5em]
            &=\frac{1-x^4}{1-x^6}\\
            &=(1-x^4)(1+x^6+x^{12}+x^{18}+\dotsc)\\
            &=1-x^4+x^6-x^{10}+x^{12}-x^{16}+\dots
    .\end{align*}

    \vspace{1em}
    and hence $\boxed{a_{3k}=1, a_{3k+1}=0, a_{3k+2}=-1}$.
}


\solu{[Combinaorial Model, MellowMelon]
    Consider the number of ways to tile a $1 \times 2n$ rectangle with squares
    and dominoes. Let $E$ be the set of ways to do it with an even number of
    dominoes, and let $O$ be the same for odd. We want to find $|E| - |O|$.\\

    Trying some basic cases, we see that the only values we get are $-1, 0,
    1$. So we want to find a pairing between the two sets that is almost a
    bijection, but doesn't work for only one element. And that bijection is:\\

    \textit{Consider the first consecutive pair of squares or domino which
    either comes at the very start or comes right after another domino. Swap
    the pair of squares to a domino or vice versa.}
}


\prob{https://artofproblemsolving.com/community/c6h1076934p5964577}{Binom 2}{}{
    \[{n \choose 0}^2-{n \choose 1}^2+{n \choose 2}^2-\dots +(-1)^n{n \choose n}^2=?\]
}

\solu{[Generating function, TinTarn]
    Denote \[f(m,n)=\sum_{k=0}^{\infty} (-1)^k \binom{n}{k} \binom{n}{m-k}\]
    We are looking for $f(n,n)$.  We have
    \begin{align*}
        \sum_{m=0}^{\infty} f(m,n) x^m &=\sum_{k=0}^{\infty} (-1)^k \binom{n}{k}
        \sum_{m=0}^{\infty} \binom{n}{m-k} x^m\\
        &=\left[\sum_{k=0}^{\infty} (-x)^k \binom{n}{k} \right]
        \left[\sum_{m=0}^{\infty} x^m \binom{n}{m} \right]\\
        &=(1-x)^n(1+x)^n\\
        &=(1-x^2)^n\\
        &=\sum_{t=0}^{\infty} (-1)^t \binom{n}{t} x^{2t}
    .\end{align*}

    Hence we see that for odd $m$ we have $f(m,n)=0$ and hence for odd $n$ we have
    $\boxed{f(n,n)=0}$.\\
    
    Now, for $n=2k$ even, we find $\boxed{f(2k,2k)=(-1)^k \binom{2k}{k}}$.
}

\solu{[Combinatorial, MellowMelon]
    Consider the number of ways to paint the squares of a $2 \times n$
    rectangle red and blue such that both rows have the same number of red
    squares. Let $E$ be the set of ways with an even number of red squares in each
    row, and let $O$ be the same for odd. We want to find $|E| - |O|$.\\

    \textit{Pair the selections as follows: take the first column with both
        squares the same color, and flip both colors. This maps elements in
        $E$ to $O$ and vice versa and is invertible, so except cases where the
    pairing is undefined, $E$ and $O$ have the same number of elements.}
}
