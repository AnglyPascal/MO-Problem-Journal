Let's play with some integer $ n $ and Fibonacci sequences $ \mod\ n $. What should we take the value of $ n $? As $ 11=16-5 $, let's take it first. We see that the period of Fibonacci sequences $ \mod\ 11 $ is at most $ 10 $. From here it is natural to make a conjecture that for prime $ n $'s, the period probably is $ n-1 $. We also design a proof that there is a sequence which doesn't contain any of $ n $'s products. \\

So let's see if it works for all primes. No it doesn't, breaks at $ 7 $. How is $ 7 $ so different than $ 11 $? The most straightforward guess is that probably $ 7\not| 4k^2-5 $ for any $ k $. And it is true. So what's so special about the primes dividing $ 4k^2-5 $?
Writing it in modular arithmetic manner, $ 4k^2\equiv 5\ (\mod\ p) $. Wait, $ 5 $ is a quadratic residue $ \mod\ p $? But isn't $ \sqrt{5} $ related to Fibonacci sequences? What's the general formula for a Fibonacci sequence starting with $ a, b $? Wait, now that explains why the period of the Fibonacci sequences $ \mod $ these primes is $ p-1 $.