\documentclass[a4paper, 11pt]{article}
\usepackage{myfile}                      

\title{\Large \textbf{\uppercase{PSet for Practice}} \\ [.5cm]\ \large\signature}


\begin{document}


	\printtitle
	{\hypersetup{hidelinks}
			
			
			\tableofcontents
	}\newpage
	



\section{China}
	
	\subsection{TST 2018}

		\prob{Let $p,q$ be positive reals with sum 1. Show that for any $n$-tuple of reals $(y_1,y_2,...,y_n)$, there exists an $n$-tuple of reals $(x_1,x_2,...,x_n)$ satisfying $$p\cdot \max\{x_i,x_{i+1}\} + q\cdot \min\{x_i,x_{i+1}\} = y_i$$for all $i=1,2,...,2017$, where $x_{2018}=x_1$.}

		\prob{A number $n$ is interesting if 2018 divides $d(n)$ (the number of positive divisors of $n$). Determine all positive integers $k$ such that there exists an infinite arithmetic progression with common difference $k$ whose terms are all interesting.}

		\prob{Circle $\omega$ is tangent to sides $AB$,$AC$ of triangle $ABC$ at $D$,$E$ respectively, such that $D\neq B$, $E\neq C$ and $BD+CE<BC$. $F$,$G$ lies on $BC$ such that $BF=BD$, $CG=CE$. Let $DG$ and $EF$ meet at $K$. $L$ lies on minor arc $DE$ of $\omega$, such that the tangent of $L$ to $\omega$ is parallel to $BC$. Prove that the incenter of $\triangle ABC$ lies on $KL$.}


		\prob{Functions $f,g:\mathbb{Z}\to\mathbb{Z}$ satisfy $$f(g(x)+y)=g(f(y)+x)$$for any integers $x,y$. If $f$ is bounded, prove that $g$ is periodic.}

		\prob{Given a positive integer $k$, call $n$ good if among $$\binom{n}{0},\binom{n}{1},\binom{n}{2},...,\binom{n}{n}$$at least $0.99n$ of them are divisible by $k$. Show that exists some positive integer $N$ such that among $1,2,...,N$, there are at least $0.99N$ good numbers.}

		\prob{Let $A_1$, $A_2$, $\cdots$, $A_m$ be $m$ subsets of a set of size $n$. Prove that $$ \sum_{i=1}^{m} \sum_{j=1}^{m}|A_i|\cdot |A_i \cap A_j|\geq \frac{1}{mn}\left(\sum_{i=1}^{m}|A_i|\right)^3.$$}



		\prob{Given a triangle $ABC$. $D$ is a moving point on the edge $BC$. Point $E$ and Point $F$ are on the edge $AB$ and $AC$, respectively, such that $BE=CD$ and $CF=BD$. The circumcircle of $\triangle BDE$ and $\triangle CDF$ intersects at another point $P$ other than $D$. Prove that there exists a fixed point $Q$, such that the length of $QP$ is constant.}

		\prob{An integer partition, is a way of writing n as a sum of positive integers. Two sums that differ only in the order of their summands are considered the same partition.

		The number of partitions of n is given by the partition function $p\left ( n \right )$. So $p\left ( 4 \right ) = 5$ .

		Determine all the positive integers so that $p\left ( n \right )+p\left ( n+4 \right )=p\left ( n+2 \right )+p\left ( n+3 \right )$.}


		\prob{Two positive integers $p,q \in \mathbf{Z}^{+}$ are given. There is a blackboard with $n$ positive integers written on it. A operation is to choose two same number $a,a$ written on the blackboard, and replace them with $a+p,a+q$. Determine the smallest $n$ so that such operation can go on infinitely.}

		\prob{Let $k, M$ be positive integers such that $k-1$ is not squarefree. Prove that there exist a positive real $\alpha$, such that $\lfloor \alpha\cdot k^n \rfloor$ and $M$ are coprime for any positive integer $n$.}


		\prob{Given positive integers $n, k$ such that $n\ge 4k$, find the minimal value $\lambda=\lambda(n,k)$ such that for any positive reals $a_1,a_2,\ldots,a_n$, we have \[ \sum\limits_{i=1}^{n} {\frac{{a}_{i}}{\sqrt{{a}_{i}^{2}+{a}_{{i}+{1}}^{2}+{\cdots}{{+}}{a}_{{i}{+}{k}}^{2}}}} \le \lambda\]Where $a_{n+i}=a_i,i=1,2,\ldots,k$}


		\prob{Let $M,a,b,r$ be non-negative integers with $a,r\ge 2$, and suppose there exists a function $f:\mathbb{Z}\rightarrow\mathbb{Z}$ satisfying the following conditions:
			(1) For all $n\in \mathbb{Z}$, $f^{(r)}(n)=an+b$ where $f^{(r)}$ denotes the composition of $r$ copies of $f$
			(2) For all $n\ge M$, $f(n)\ge 0$
			(3) For all $n>m>M$, $n-m|f(n)-f(m)$\\
		Show that $a$ is a perfect $r$-th power.}


		\prob{Let $\omega_1,\omega_2$ be two non-intersecting circles, with circumcenters $O_1,O_2$ respectively, and radii $r_1,r_2$ respectively where $r_1 < r_2$. Let $AB,XY$ be the two internal common tangents of $\omega_1,\omega_2$, where $A,X$ lie on $\omega_1$, $B,Y$ lie on $\omega_2$. The circle with diameter $AB$ meets $\omega_1,\omega_2$ at $P$ and $Q$ respectively. If $$\angle AO_1P+\angle BO_2Q=180^{\circ},$$find the value of $\frac{PX}{QY}$ (in terms of $r_1,r_2$).}


		\prob{Let $G$ be a simple graph with 100 vertices such that for each vertice $u$, there exists a vertice $v \in N \left ( u \right )$ and $ N \left ( u \right ) \cap  N \left ( v \right ) = \o $. Try to find the maximal possible number of edges in $G$. The $ N \left ( . \right )$ refers to the neighborhood.}


		\prob{Prove that there exists a constant $C>0$ such that
		$$H(a_1)+H(a_2)+\cdots+H(a_m)\leq C\sqrt{\sum_{i=1}^{m}i a_i}$$holds for arbitrary positive integer $m$ and any $m$ positive integer $a_1,a_2,\cdots,a_m$, where $$H(n)=\sum_{k=1}^{n}\frac{1}{k}.$$}


		\prob{Suppose $A_1,A_2,\cdots ,A_n \subseteq \left \{ 1,2,\cdots ,2018 \right \}$ and $\left | A_i \right |=2, i=1,2,\cdots ,n$, satisfying that $$A_i + A_j, \; 1 \le i \le j \le n ,$$are distinct from each other. $A + B = \left \{ a+b|a\in A,\,b\in B \right \}$. Determine the maximal value of $n$.}


		\prob{Let $ABC$ be a triangle with $\angle BAC > 90 ^{\circ}$, and let $O$ be its circumcenter and $\omega$ be its circumcircle. The tangent line of $\omega$ at $A$ intersects the tangent line of $\omega$ at $B$ and $C$ respectively at point $P$ and $Q$. Let $D,E$ be the feet of the altitudes from $P,Q$ onto $BC$, respectively. $F,G$ are two points on $\overline{PQ}$ different from $A$, so that $A,F,B,E$ and $A,G,C,D$ are both concyclic. Let M be the midpoint of $\overline{DE}$. Prove that $DF,OM,EG$ are concurrent.}


		\prob{Find all pairs of positive integers $(x, y)$ such that $(xy+1)(xy+x+2)$ be a perfect square .}


		\prob{Define the polymonial sequence $\left \{ f_n\left ( x \right ) \right \}_{n\ge 1}$ with $f_1\left ( x \right )=1$, $$f_{2n}\left ( x \right )=xf_n\left ( x \right ), \; f_{2n+1}\left ( x \right ) = f_n\left ( x \right )+ f_{n+1} \left ( x \right ), \; n\ge 1.$$Look for all the rational number $a$ which is a root of certain $f_n\left ( x \right ).$}


		\prob{There are $32$ students in the class with $10$ interesting group. Each group contains exactly $16$ students. For each couple of students, the square of the number of the groups which are only involved by just one of the two students is defined as their $interests-disparity$. Define $S$ as the sum of the $interests-disparity$ of all the couples, $\binom{32}{2}\left ( =\: 496 \right )$ ones in total. Determine the minimal possible value of $S$.}


		\prob{In isosceles $\triangle ABC$, $AB=AC$, points $D,E,F$ lie on segments $BC,AC,AB$ such that $DE\parallel AB$, $DF\parallel AC$. The circumcircle of $\triangle ABC$ $\omega_1$ and the circumcircle of $\triangle AEF$ $\omega_2$ intersect at $A,G$. Let $DE$ meet $\omega_2$ at $K\neq E$. Points $L,M$ lie on $\omega_1,\omega_2$ respectively such that $LG\perp KG, MG\perp CG$. Let $P,Q$ be the circumcenters of $\triangle DGL$ and $\triangle DGM$ respectively. Prove that $A,G,P,Q$ are concyclic.}


		\prob{Let $p$ be a prime and $k$ be a positive integer. Set $S$ contains all positive integers $a$ satisfying $1\le a \le p-1$, and there exists positive integer $x$ such that $x^k\equiv a \pmod p$.

		Suppose that $3\le |S| \le p-2$. Prove that the elements of $S$, when arranged in increasing order, does not form an arithmetic progression.}


		\prob{Suppose the real number $\lambda \in \left( 0,1\right),$ and let $n$ be a positive integer. Prove that the modulus of all the roots of the polynomial $$f\left ( x \right )=\sum_{k=0}^{n}\binom{n}{k}\lambda^{k\left ( n-k \right )}x^{k}$$are $1.$}


		\prob{Suppose $a_i, b_i, c_i, i=1,2,\cdots ,n$, are $3n$ real numbers in the interval $\left [ 0,1 \right ].$ Define $$S=\left \{ \left ( i,j,k \right ) |\, a_i+b_j+c_k<1 \right \}, \; \; T=\left \{ \left ( i,j,k \right ) |\, a_i+b_j+c_k>2 \right \}.$$Now we know that $\left | S \right |\ge 2018,\, \left | T \right |\ge 2018.$ Try to find the minimal possible value of $n$.}



	\subsection{TST 2017}

		\prob{Find out the maximum value of the numbers of edges of a solid regular octahedron that we can see from a point out of the regular octahedron.(We define we can see an edge $AB$ of the regular octahedron from point $P$ outside if and only if the intersection of non degenerate triangle $PAB$ and the solid regular octahedron is exactly edge $AB$.}

		\prob{Let $x>1$ ,$n$ be positive integer. Prove that$$\sum_{k=1}^{n}\frac{\{kx \}}{[kx]}<\sum_{k=1}^{n}\frac{1}{2k-1}$$Where $[kx ]$ be the integer part of $kx$ ,$\{kx \}$ be the decimal part of $kx$.}

		\prob{Suppose $S=\{1,2,3,...,2017\}$,for every subset $A$ of $S$,define a real number $f(A)\geq 0$ such that:
			$(1)$ For any $A,B\subset S$,$f(A\cup B)+f(A\cap B)\leq f(A)+f(B)$;
			$(2)$ For any $A\subset B\subset S$, $f(A)\leq f(B)$;
			$(3)$ For any $k,j\in S$,$$f(\{1,2,\ldots,k+1\})\geq f(\{1,2,\ldots,k\}\cup \{j\});$$$(4)$ For the empty set $\varnothing$, $f(\varnothing)=0$.
		Confirm that for any three-element subset $T$ of $S$,the inequality $$f(T)\leq \frac{27}{19}f(\{1,2,3\})$$holds.}

		\prob{Find out all the integer pairs $(m,n)$ such that there exist two monic polynomials $P(x)$ and $Q(x)$ ,with $\deg{P}=m$ and $\deg{Q}=n$,satisfy that $$P(Q(t))\not=Q(P(t))$$holds for any real number $t$.}

		\prob{In the non-isosceles triangle $ABC$,$D$ is the midpoint of side $BC$,$E$ is the midpoint of side $CA$,$F$ is the midpoint of side $AB$.The line(different from line $BC$) that is tangent to the inscribed circle of triangle $ABC$ and passing through point $D$ intersect line $EF$ at $X$.Define $Y,Z$ similarly.Prove that $X,Y,Z$ are collinear.}

		\prob{For a given positive integer $n$ and prime number $p$, find the minimum value of positive integer $m$ that satisfies the following property: for any polynomial $$f(x)=(x+a_1)(x+a_2)\ldots(x+a_n)$$($a_1,a_2,\ldots,a_n$ are positive integers), and for any non-negative integer $k$, there exists a non-negative integer $k'$ such that $$v_p(f(k))<v_p(f(k'))\leq v_p(f(k))+m.$$Note: for non-zero integer $N$,$v_p(N)$ is the largest non-zero integer $t$ that satisfies $p^t\mid N$.}

		\prob{Let $n$ be a positive integer. Let $D_n$ be the set of all divisors of $n$ and let $f(n)$ denote the smallest natural $m$ such that the elements of $D_n$ are pairwise distinct in mod $m$. Show that there exists a natural $N$ such that for all $n \geq N$, one has $f(n) \leq n^{0.01}$.}

		\prob{$2017$ engineers attend a conference. Any two engineers if they converse, converse with each other in either Chinese or English. No two engineers converse with each other more than once. It is known that within any four engineers, there was an even number of conversations and furthermore within this even number of conversations:

		i) At least one conversation is in Chinese.
		ii) Either no conversations are in English or the number of English conversations is at least that of Chinese conversations.

		Show that there exists $673$ engineers such that any two of them conversed with each other in Chinese.}

		\prob{Let $ABCD$ be a quadrilateral and let $l$ be a line. Let $l$ intersect the lines $AB,CD,BC,DA,AC,BD$ at points $X,X',Y,Y',Z,Z'$ respectively. Given that these six points on $l$ are in the order $X,Y,Z,X',Y',Z'$, show that the circles with diameter $XX',YY',ZZ'$ are coaxal.}

		\prob{An integer $n>1$ is given . Find the smallest positive number $m$ satisfying the following conditions: for any set $\{a,b\}$ $\subset \{1,2,\cdots,2n-1\}$ ,there are non-negative integers $ x, y$ ( not all zero) such that $2n|ax+by$ and $x+y\leq m.$}


		\prob{Let $ \varphi(x)$ be a cubic polynomial with integer coefficients. Given that $ \varphi(x)$ has have 3 distinct real roots $u,v,w $ and $u,v,w $ are not rational number. there are integers $ a, b,c$ such that $u=av^2+bv+c$. Prove that $b^2 -2b -4ac - 7$ is a square number .}


		\prob{Let $M$ be a subset of $\mathbb{R}$ such that the following conditions are satisfied:

			a) For any $x \in M, n \in \mathbb{Z}$, one has that $x+n \in \mathbb{M}$.
			b) For any $x \in M$, one has that $-x \in M$.
			c) Both $M$ and $\mathbb{R}$ \ $M$ contain an interval of length larger than $0$.

		For any real $x$, let $M(x) = \{ n \in \mathbb{Z}^{+} | nx \in M \}$. Show that if $\alpha,\beta$ are reals such that $M(\alpha) = M(\beta)$, then we must have one of $\alpha + \beta$ and $\alpha - \beta$ to be rational.}


		\prob{Let $n \geq 4$ be a natural and let $x_1,\ldots,x_n$ be non-negative reals such that $x_1 + \cdots + x_n = 1$. Determine the maximum value of $x_1x_2x_3 + x_2x_3x_4 + \cdots + x_nx_1x_2$.}


		\prob{Let $ABCD$ be a non-cyclic convex quadrilateral. The feet of perpendiculars from $A$ to $BC,BD,CD$ are $P,Q,R$ respectively, where $P,Q$ lie on segments $BC,BD$ and $R$ lies on $CD$ extended. The feet of perpendiculars from $D$ to $AC,BC,AB$ are $X,Y,Z$ respectively, where $X,Y$ lie on segments $AC,BC$ and $Z$ lies on $BA$ extended. Let the orthocenter of $\triangle ABD$ be $H$. Prove that the common chord of circumcircles of $\triangle PQR$ and $\triangle XYZ$ bisects $BH$.}


		\prob{Let $X$ be a set of $100$ elements. Find the smallest possible $n$ satisfying the following condition: Given a sequence of $n$ subsets of $X$, $A_1,A_2,\ldots,A_n$, there exists $1 \leq i < j < k \leq n$ such that
		$$A_i \subseteq A_j \subseteq A_k \text{ or } A_i \supseteq A_j \supseteq A_k.$$}


		\prob{Show that there exists a degree $58$ monic polynomial
		$$P(x) = x^{58} + a_1x^{57} + \cdots + a_{58}$$such that $P(x)$ has exactly $29$ positive real roots and $29$ negative real roots and that $\log_{2017} |a_i|$ is a positive integer for all $1 \leq i \leq 58$.}


		\prob{Show that there exists a positive real $C$ such that for any naturals $H,N$ satisfying $H \geq 3, N \geq e^{CH}$, for any subset of $\{1,2,\ldots,N\}$ with size $\lceil \frac{CHN}{\ln N} \rceil$, one can find $H$ naturals in it such that the greatest common divisor of any two elements is the greatest common divisor of all $H$ elements.}


		\prob{Every cell of a $2017\times 2017$ grid is colored either black or white, such that every cell has at least one side in common with another cell of the same color. Let $V_1$ be the set of all black cells, $V_2$ be the set of all white cells. For set $V_i (i=1,2)$, if two cells share a common side, draw an edge with the centers of the two cells as endpoints, obtaining graphs $G_i$. If both $G_1$ and $G_2$ are connected paths (no cycles, no splits), prove that the center of the grid is one of the endpoints of $G_1$ or $G_2$.}


		\prob{Prove that :$$\sum_{k=0}^{58}C_{2017+k}^{58-k}C_{2075-k}^{k}=\sum_{p=0}^{29}C_{4091-2p}^{58-2p}$$}


		\prob{In $\varDelta{ABC}$,the excircle of $A$ is tangent to segment $BC$,line $AB$ and $AC$ at $E,D,F$ respectively.$EZ$ is the diameter of the circle.$B_1$ and $C_1$ are on $DF$, and $BB_1\perp{BC}$,$CC_1\perp{BC}$.Line $ZB_1,ZC_1$ intersect $BC$ at $X,Y$ respectively.Line $EZ$ and line $DF$ intersect at $H$,$ZK$ is perpendicular to $FD$ at $K$.If $H$ is the orthocenter of $\varDelta{XYZ}$,prove that:$H,K,X,Y$ are concyclic.}


		\prob{Find the numbers of ordered array $(x_1,...,x_{100})$ that satisfies the following conditions:
			($i$)$x_1,...,x_{100}\in\{1,2,..,2017\}$;
			($ii$)$2017|x_1+...+x_{100}$;
			($iii$)$2017|x_1^2+...+x_{100}^2$.}


		\prob{Given integer $d>1,m$,prove that there exists integer $k>l>0$, such that $$(2^{2^k}+d,2^{2^l}+d)>m.$$}


		\prob{Given integer $m\geq2$,$x_1,...,x_m$ are non-negative real numbers,prove that:$$(m-1)^{m-1}(x_1^m+...+x_m^m)\geq(x_1+...+x_m)^m-m^mx_1...x_m$$and please find out when the equality holds.}


		\prob{A plane has no vertex of a regular dodecahedron on it,try to find out how many edges at most may the plane intersect the regular dodecahedron?}


		\prob{Given $n\ge 3$. consider a sequence $a_1,a_2,...,a_n$, if $(a_i,a_j,a_k)$ with i+k=2j (i<j<k) and $a_i+a_k\ne 2a_j$, we call such a triple a $NOT-AP$ triple. If a sequence has at least one $NOT-AP$ triple, find the least possible number of the $NOT-AP$ triple it contains.}


		\prob{Find the least positive number m such that for any polynimial f(x) with real coefficients, there is a polynimial g(x) with real coefficients (degree not greater than m) such that there exist 2017 distinct number $a_1,a_2,...,a_{2017}$ such that $g(a_i)=f(a_{i+1})$ for i=1,2,...,2017 where indices taken modulo 2017.}


		\prob{For a rational point (x,y), if xy is an integer that divided by 2 but not 3, color (x,y) red, if xy is an integer that divided by 3 but not 2, color (x,y) blue. Determine whether there is a line segment in the plane such that it contains exactly 2017 blue points and 58 red points.}


		\prob{Given a circle with radius 1 and 2 points C, D given on it. Given a constant l with $0<l\le 2$. Moving chord of the circle AB=l and ABCD is a non-degenerated convex quadrilateral. AC and BD intersects at P. Find the loci of the circumcenters of triangles ABP and BCP.}


		\prob{A(x,y), B(x,y), and C(x,y) are three homogeneous real-coefficient polynomials of x and y with degree 2, 3, and 4 respectively. we know that there is a real-coefficient polinimial R(x,y) such that $B(x,y)^2-4A(x,y)C(x,y)=-R(x,y)^2$. Proof that there exist 2 polynomials F(x,y,z) and G(x,y,z) such that $F(x,y,z)^2+G(x,y,z)^2=A(x,y)z^2+B(x,y)z+C(x,y)$ if for any x, y, z real numbers $A(x,y)z^2+B(x,y)z+C(x,y)\ge 0$}


		\prob{We call a graph with n vertices $k-flowing-chromatic$ if:
			1. we can place a chess on each vertex and any two neighboring (connected by an edge) chesses have different colors.
			2. we can choose a hamilton cycle $v_1,v_2,\cdots , v_n$, and move the chess on $v_i$ to $v_{i+1}$ with $i=1,2,\cdots ,n$ and $v_{n+1}=v_1$, such that any two 			neighboring chess also have different colors.
			3. after some action of step 2 we can make all the chess reach each of the n vertices.
			Let T(G) denote the least number k such that G is k-flowing-chromatic.
			If such k does not exist, denote T(G)=0.
			denote $\chi (G)$ the chromatic number of G.
			Find all the positive number m such that there is a graph G with $\chi (G)\le m$ and $T(G)\ge 2^m$ without a cycle of length small than 2017.}



	\subsection{MO 2018}


		\prob{Let $a\le 1$ be a real number. Sequence $\{x_n\}$ satisfies $x_0=0, x_{n+1}= 1-a\cdot e^{x_n}$, for all $n\ge 1$, where $e$ is the natural logarithm. Prove that for any natural $n$, $x_n\ge 0$.}


		\prob{Points $D,E$ lie on segments $AB,AC$ of $\triangle ABC$ such that $DE\parallel BC$. Let $O_1,O_2$ be the circumcenters of $\triangle ABE, \triangle ACD$ respectively. Line $O_1O _2$ meets $AC$ at $P$, and $AB$ at $Q$. Let $O$ be the circumcenter of $\triangle APQ$, and $M$ be the intersection of $AO$ extended and $BC$. Prove that $M$ is the midpoint of $BC$.}


		\prob{Given a real sequence $\left \{ x_n \right \}_{n=1}^{\infty}$ with $x_1^2 = 1$. Prove that for each integer $n \ge 2$, $$\sum_{i|n}\sum_{j|n}\frac{x_ix_j}{\textup{lcm} \left ( i,j \right )} \ge \prod_{\mbox{\tiny$\begin{array}{c} p \: \textup{is prime} \\ p|n \end{array}$} }\left ( 1-\frac{1}{p} \right ). $$}


		\prob{There're $n$ students whose names are different from each other. Everyone has $n-1$ envelopes initially with the others' name and address written on them respectively. Everyone also has at least one greeting card with her name signed on it. Everyday precisely a student encloses a greeting card (which can be the one received before) with an envelope (the name on the card and the name on envelope cannot be the same) and post it to the appointed student by a same day delivery.
			Prove that when no one can post the greeting cards in this way any more:
			(i) Everyone still has at least one card;
			(ii) If there exist $k$ students $p_1, p_2, \cdots, p_k$ so that $p_i$ never post a card to $p_{i+1}$, where $i = 1,2, \cdots, k$ and $p_{k+1} = p_1$, then these $k$ students have prepared the same number of greeting cards initially.}


		\prob{Let $\omega \in \mathbb{C}$, and $\left |  \omega  \right | = 1$. Find the maximum length of $z = \left( \omega + 2 \right) ^3 \left( \omega - 3 \right)^2$.}



		\prob{Given $k \in \mathbb{N}^+$. A sequence of subset of the integer set $\mathbb{Z} \supseteq I_1 \supseteq I_2 \supseteq \cdots \supseteq I_k$ is called a $k-chain$ if for each $1 \le i \le k$ we have
			(i) $168 \in I_i$;
			(ii) $\forall x, y \in I_i$, we have $x-y \in I_i$.
			Determine the number of $k-chain$ in total.}


		\prob{Given $2018 \times 4$ grids and tint them with red and blue. So that each row and each column has the same number of red and blue grids, respectively. Suppose there're $M$ ways to tint the grids with the mentioned requirement. Determine $M \pmod {2018}$.}


		\prob{Let $I$ be the incenter of triangle $ABC$. The tangent point of $\odot I$ on $AB,AC$ is $D,E$, respectively. Let $BI \cap AC = F$, $CI \cap AB = G$, $DE \cap BI = M$, $DE \cap CI = N$, $DE \cap FG = P$, $BC \cap IP = Q$. Prove that $BC = 2MN$ is equivalent to $IQ = 2IP$.}



	\subsection{MO 2017}

	
\end{document}