\documentclass{article}
\vspace{10cm}
\title{\huge \textbf{Geometry Problem Set}\\[0.3cm]
	\Large National Camp 2018}
\date{}
\author{Asif E Elahi, M. Ahsan Al Mahir\thanks{Originally by Asif E Elahi, later modified and enhanced by M Ahsan Al Mahir}}

\usepackage{header_asif_vai_geo}


\begin{document}
	\maketitle
	
\section{Basic Stuffs}
	
	The problems that are listed below are your tools for solving tougher olympiad problems, be sure to know these by heart.
	
%1
	\prob{	Prove that the diagonals of a rhombus are perpendicular. }
	
%2
	\prob{	Let $L,M$ be the midpoints of $BC$ and $CA$ of $\triangle ABC$ respectively. Prove that $AL=BM \Longleftrightarrow AC=BC$.	}
	
%3
	\prob{  Let $P,Q,R,S$ be four points on a plane. Prove that \footnote{This is often called \textbf{Perpendicularity Lemma} in olympiad folklore}
		$PR \perp QS \Longleftrightarrow PQ^2-QR^2=PS^2-RS^2$. 	}
	
%4

	\prob{  Let the circles $\omega_1$ and $\omega_2$ meet at $X, Y$. Two lines $l_1, l_2$ through $X$ intersect $\omega_1, \omega_2$ at $P_1, P_2$ and $Q_1, Q_2$ respectively. Prove that $\triangle YP_1Q_1$ and $\triangle YP_2Q_2$ are similar. }
	
	\textbf{Note:} This little and easy problem might seem very trivial, but this can be very useful in dealing with harder problems. Yufei Zhao's 3 lemmas in geometry for further reading.
	
	
%5
	\prob{
		
		\begin{enumerate}
			
			\item Prove that for all $ \triangle ABC $ the following relations are true:
			
			\[\frac{a}{\sin A}=\frac{b}{\sin B}=\frac{c}{\sin C}=2R\] ($ R $ is the circumradius)
			
			\item In $\triangle ABC$, $P$ lies on $BC$. Prove that \footnote{This is a very important lemma!} 	
			
			\[\frac{BP}{CP}=\frac{AB\times sin\angle BAP}{AC \times sin \angle PAC}\]
			
	\end{enumerate}		}
	
	
%6
	\prob{	Let $P$ and $Q$ be arbitrary points on sides $BC$ and $CA$ respectively. Let the internal bisectors of $\angle CAP$ and $\angle CBQ$ meet at $R$. Prove that $\angle AQB+ \angle APB=2\angle ARB$.  }


%7	
	\prob{	Let $P,Q,R$ be points on sides $BC,CA,AB$ of $\triangle ABC$. Prove that the perpendiculars to the sides at these points are concurrent if and only if $BP^2+CQ^2+AR^2=PC^2+QA^2+RB^2$.    }
	
%8	
	\prob{	Let $D,E,F$ are the midpoints of $BC,CA,AB$ resp. Prove that $\angle CAD=\angle ABE\Longleftrightarrow \angle AFC=\angle ADB$. }
	
	
%9
	\prob{	Let the angle bisector of $\angle BAC$ meets $\bigodot ABC$ at $A$ and $X$ resp. Prove that $XI=XB=XC=XI_a$ where $I$ is the incenter and $I_a$ is the excenter opposite to $A$ of $\triangle ABC$.	}
	
	\textbf{Note:} This is important as well.
	
	
%10

	\prob{	Let circles $S_1$ and $S_2$ meet at points $A$ and $B$. An arbitrary line passing through $A$ intersects $S_1$ and $S_2$ at $P$ and $Q$ resp. Prove $\dfrac{BP}{BQ}$ is constant.	}
	
	
%11
	\prob{	Let $L,M,N$ are the midpoints of $BC,CA,AB$ and $AD,BE,CF$ are altitudes of $\triangle ABC$. Prove that
	\begin{itemize}
		\item $O$ is the orthocenter of $\triangle LMN$.
		\item $H$ is the incenter of $\triangle DEF$.
		\item $D,E,F,L,M,N$ all lie on a circle.
		\item The center of this circle is the midpoint of $OH$.
		\item Let $BO \cap \bigodot ABC=Q$. Prove that $AQCH$ is a parallelogram
		\item Prove that $ AH=a\cot A=2R\cos A $ ($ R $ is the circumradius) and $ HD=2\cos B\cos C $
		\item Prove that the reflection of $ H $ on $ BC $ lies on the circumcenter.
		\item Prove that the reflection of the \textbf{Euler Line}\footnote{It is the line joining the orthocenter and the circumcenter} on the sides of $ \triangle ABC $
		concur at the circumcirle.
			
	\end{itemize}}
	
%12
	\prob{	In $\triangle ABC$, $\angle BAC=90^{\circ}$, $AD$ is an altitude. The circle with center $A$ and radius $AD$ meets $\bigodot ABC$ at $U$ and $V$. Prove that $UV$ passes through the midpoint of $AD$.	}
	
%13
	\prob{	Let the incircle and excircle (opposite to $A$) of $\triangle ABC$ meet $BC$ at $D$ and $E$ resp. Suppose $F$ is the antipode of $D$ wrt the incircle.
		
	\begin{enumerate}
		\item Prove that $A,F,E$ are collinear.
		\item  $M$ be the midpoint of $DE$. Prove that $MI$ meets $AD$ at it's midpoint. 
	\end{enumerate}}
	
%14
	\prob{	Let the incircle of $\triangle ABC$ meets $AB$ and $AC$ at $X$ and $Y$ resp. $BI$ and $CI$ meet $XY$ at $P$ and $Q$ respectively. Prove that $BPQC$ is cyclic. (In fact $BP\perp CP$ and $BQ\perp CQ$)}
	
%15
	\prob{	If four points $A,C,B,D$ lie on a line in this order satisfying the property that $\dfrac{AC}{BC}=\dfrac{AD}{BD}$, then $A,B,C,D$ are in harmonic order. Prove that if  $A,B,C,D$ are in harmonic order and $M$ is the midpoint of $AB$, then
		
	\begin{enumerate}
		\item $MA^2=MC.MD$ and $DA.DB=DC.DM$.
		\item If $P$ is a point s.t $\angle APB=90^{\circ}$, then $PA$ and $PB$ are two bisectors of $\angle CPD$.
		\item Suppose $Q$ is point in the plane. Let a line $l$ meets $QA,QB,QC,QD$ at four points $A_1,B_1,C_1,D_1$ respectively. Then prove that $A_1,B_1,C_1,D_1$ are also in harmonic order.
	\end{enumerate}}
	
	\textbf{Note:} This is the one of the most important lemma or theorem what you may call it, in bamming projective problems. For further reading go to Alexander Remorov's Projective Geometry handout.
	
%16
	\prob{	$AD$ is an altitude of $\triangle ABC$. $E,F$ are on $AC,AB$ so that $AD,BE,CF$ are concurrent. Prove $\angle EDA=\angle FDA$.	}
	
%17
	\prob{	Let $AD$ be an altitude of $\triangle ABC$ and $E\in \bigodot ABC$ so that $AE\parallel BC$. Prove that $D,G,E$ are collinear where $G$ is the centroid of $\triangle ABC$.	}
	
%18
	\prob{	Let $O$ be the circumcenter of $\triangle ABC$ and $A',B',C'$ are reflections of $A$ on $BC,CA,AB$ resp. Prove that $AA',BB',CC'$ are concurrent.	}
	
%19
	\prob{	Let $D,E$ are on sides $AC,AB$ of $\triangle ABC$ resp. such that $BE=CD$. Let $\bigodot ABC \cap \bigodot ADE= P$. Prove that $PB=PC$.	}
	
%20
	\prob{	Let a line $PQ$ touch circle $S_1$ and $S_2$ at $P$ and $Q$ resp. Prove that the radical axis of $S_1$ and $S_2$ passes through the midpoint of $PQ$.	}
	
%21
	\prob{	Let $\omega_1,\omega_2,omega_3$ are $3$ circles. Prove that the $3$ radical axis of $\omega_1$ and $\omega_2$,$\omega_2$ and $\omega_3$,$\omega_3$ and $\omega_1$ are either concurrent or parallel. 	}
	
%22
	\prob{	 Two equal-radius circles $\omega_1$ and $\omega_2$ are centered at points $O_1$ and $O_2$ . A point $X$ is reflected through $O_1$ and $O_2$ to get points $A_1$ and $A_2$. 
		The tangents from $A_1$ to $\omega_1$ touch $\omega_1$ at points $P_1$ and $Q_1$, and 
		the tangents from $A_2$ to $\omega_2$ touch $\omega_2$ at points $P_2$ and $Q_2$. 
		If $P_1Q_1$ and $P_2Q_2$ intersect at $Y$ , prove that $Y$ is equidistant from $A_1$ and $A_2$.  	}
	
%23
	\prob{	Let $BD,CE$ be the altitudes of $\triangle ABC$ and $M$ be the midpoint of $BC$. If the ray $MH$ meet $\bigodot ABC$ at point $K$, prove that $AK,BC,DE$ are concurrent. 	}
	
%24
	\prob{	Two circle $\omega$ and $\Gamma$ touches one another internally at $P$ with $\omega $ inside of $\Gamma$. Let $AB$ be a chord of $\Gamma$ which touches $\omega$ at $D$. Let $PD\cap \Gamma=Q$. Prove that $QA=QB$.	}
	
%25
	\prob{	Let $AD$ be a symmedian of $\triangle ABC$ with $D$ on $\bigodot ABC$. Let $M$ be the midpoint of $AD$. Prove that $\angle BMD=\angle CMD$ and $A,M,O,D$ are cyclic where $O$ is the circumcenter of $\triangle ABC$. 	}
	
%26
	\prob{	Let $ A, B $ be two fixed points and let $ P $ be varying point such that $ \frac{PA}{PB} $ is constant. Prove that the locus of $ P $ is a circle.	}
	
%27
	\prob{ Prove that $ r_1+r_2+r_3=4R+r $ ($ R, r, r_1, r_2, r_3 $ are the circumradius, inradius and three exradiuses respectively of a triangle)}
	
%28
	\prob{	Let $ M $  be the midpoint of the altitude $ BE $ in $ \triangle ABC $ and suppose that the excircle opposite to $ B $ touches $ AC $ at $ Y $. Then $ MY $ goes through the incenter $ I $.}
	
%29
	\prob{	Let $ ABC $ be a triangle, and draw isosceles triangles $ \triangle DBC, \triangle AEC, \triangle ABF $
external to $ \triangle ABC $ (with $ BC; CA; AB $ as their respective bases). Prove that the lines through
$ A; B; C $ perpendicular to $ EF; FD; DE $, respectively, are concurrent.}

%30
	\prob{	In a triangle $ ABC $ we have $ AB = AC $. A circle which is internally tangent
with the circumscribed circle of the triangle is also tangent to the sides $ AB; AC $ in the points $ P $, respectively $ Q $. Prove that the midpoint of $ PQ $ is the center of the inscribed circle of the triangle
$ ABC $}
	
%31
	\prob{\textbf{Nagel Point $ N $}: If the Excircles of $ ABC $ touch $ BC; CA; AB $ at $ D; E; F $, then
the intersection point of $ AD; BE; CF $ is called the \textbf{Nagel Point} $ N $. Prove that
		
	\begin{enumerate}
		\item $ I; G; N $ are collinear. ($ G $ centroid, $ I $ incenter.)
		\item $ GN = 2 \cdot IG $.
		\item \textbf{Speiker center} $ S $: The incircle of the medial triangle is called the Speiker circle, and it’s center is \textbf{Speiker center} $ S $. Prove that $ S $ is the midpoint of $ IN $.
	\end{enumerate}}

	
	
\bigskip\section{Olympiad Problems}

The problems below are not sorted by difficulty. These are really nice problems, so try all of them :)
	
	
%1
	\prob{	Let $PB$ and $PC$ are tangent to $\bigodot ABC$. Let $D,E,F$ are projection of $A$ on $BC,PB,PC$ resp. Prove that $AD^2=AE\times AF$.	}
	
%2
	\prob{	Let $D$ and $E$ are on $AB$ and $AC$ s.t $DE \parallel BC$. $P$ is an arbitrary point inside $\triangle ADE$. $PB,PC\cap DE=F,G$. Let $\bigodot PDG \cap \bigodot PFE=Q$. Prove that $A,P,Q$ are collinear.  	}
	
%3
	\prob{	Let $AB$ and $CD$ be chords in a circle of center $O$ with $A , B , C , D$ distinct , and with the lines $AB$ and $CD$ meeting at a right angle at point $E$. Let also $M$ and $N$ be the midpoints of $AC$ and $BD$ respectively . If $MN \bot OE$ , prove that $AD \parallel BC$	}
	
%4
	\prob{	Circles $ \mathcal{C}_1$ and $ \mathcal{C}_2$ intersect at $ A$ and $ B$. Let $ M\in AB$. A line through $ M$ (different from $ AB$) cuts circles $ \mathcal{C}_1$ and $ \mathcal{C}_2$ at $ Z,D,E,C$ respectively such that $ D,E\in ZC$. Perpendiculars at $ B$ to the lines $ EB,ZB$ and $ AD$ respectively cut circle $ \mathcal{C}_2$ in $ F,K$ and $ N$. Prove that $ KF=NC$. }
	
%5
	\prob{	Let $D$ be a point on side $AC$ of triangle $ABC$. Let $E$ and $F$ be points on the segments $BD$ and $BC$ respectively, such that $\angle BAE = \angle CAF$. Let $P$ and $Q$ be points on $BC$ and $BD$ respectively, such that $EP$ and $FQ$ are both parallel to $CD$. Prove that $\angle BAP = \angle CAQ$.	}
	
%6
	\prob{	In the non-isosceles triangle $ABC$ an altitude from $A$ meets side $BC$ in $D$ . Let $M$ be the midpoint of $BC$ and let $N$ be the reflection of $M$ in $D$. The circumcirle of triangle $AMN$ intersects the side $AB$ in $P\ne A$ and the side $AC$ in $Q\ne A$ . Prove that $AN,\ BQ$ and $CP$ are concurrent.	}
	
%7
	\prob{	In triangle $ABC$, the interior and exterior angle bisectors of $ \angle BAC$ intersect the line $BC$ in $D $ and $E$, respectively. Let $F$ be the second point of intersection of the line $AD$ with the circumcircle of the triangle $ ABC$. Let $O$ be the circumcenter of the triangle $ ABC $and let $D'$ be the reflection of $D$ in $O$. Prove that $ \angle D'FE =90.$	}
	
%8
	\prob{	Let $ABCD$ be a convex quadrilateral such that the line $BD$ bisects the angle $ABC.$ The circumcircle of triangle $ABC$ intersects the sides $AD$ and $CD$ in the points $P$ and $Q,$ respectively. The line through $D$ and parallel to $AC$ intersects the lines $BC$ and $BA$ at the points $R$ and $S,$ respectively. Prove that the points $P, Q, R$ and $S$ lie on a common circle.	}
	
%9
	\prob{	   The incircle of triangle $ABC$ touches $BC$, $CA$, $AB$ at points $A_1$, $B_1$, $C_1$, respectively. The perpendicular from the incenter $I$ to the median from vertex $C$ meets the line $A_1B_1$ in point $K$. Prove that $CK$ is parallel to $AB$.	}
	
%10
	\prob{	   Let $X$ be an arbitrary point inside the circumcircle of a triangle $ABC$. The lines $BX$ and $CX$ meet the circumcircle in points $K$ and $L$ respectively. The line $LK$ intersects $BA$ and $AC$ at points $E$ and $F$ respectively. Find the locus of points $X$ such that the circumcircles of triangles $AFK$ and $AEL$ touch.	}
	
%11
	\prob{	Let $BD$ be a bisector of triangle $ABC$. Points $I_a$, $I_c$ are the incenters of triangles $ABD$, $CBD$ respectively. The line $I_aI_c$ meets $AC$ in point $Q$. Prove that $\angle DBQ = 90^\circ$.	}
	
%12
	\prob{	   Given right-angled triangle $ABC$ with hypotenuse $AB$. Let $M$ be the midpoint of $AB$ and $O$ be the center of circumcircle $\omega$ of triangle $CMB$. Line $AC$ meets $\omega$ for the second time in point $K$. Segment $KO$ meets the circumcircle of triangle $ABC$ in point $L$. Prove that segments $AL$ and $KM$ meet on the circumcircle of triangle $ACM$.	}
	
%13
	\prob{	 Let $BN$ be median of triangle $ABC$. $M$ is a point on $BC$. $S$ lies on $BN$ such that $MS\parallel AB$. $P$ is a point such that $SP\perp AC$ and $BP\parallel AC$. $MP$ cuts $AB$ at $Q$. Prove that $QB=QP$.	}
	
%14
	\prob{	    Let $ABCD$ be a convex quadrilateral with $AB$ parallel to $CD$. Let $P$ and $Q$ be the midpoints of $AC$ and $BD$, respectively. Prove that if $\angle ABP=\angle CBD$, then $\angle BCQ=\angle ACD$.		}
	
%15
	\prob{	Point $P$ lies inside a triangle $ABC$. Let $D,E$ and $F$ be reflections of the point $P$ in the lines $BC,CA$ and $AB$, respectively. Prove that if the triangle $DEF$ is equilateral, then the lines $AD,BE$ and $CF$ intersect in a common point.	}
	
%16
	\prob{	Let $\triangle{ABC}$ be an acute angled triangle. The circle with diameter $AB$ intersects the sides $AC$ and $BC$ at points $E$ and $F$ respectively. The tangents drawn to the circle through $E$ and $F$ intersect at $P$. Show that $P$ lies on the altitude through the vertex $C$.	}
	
%17
	\prob{	Let $\gamma$ be circle and let $P$ be a point outside $\gamma$. Let $PA$ and $PB$ be the tangents from $P$ to $\gamma$ (where $A, B \in \gamma$). A line passing through $P$ intersects $\gamma$ at points $Q$ and $R$. Let $S$ be a point on $\gamma$ such that $BS \parallel QR$. Prove that $SA$ bisects $QR$	}
	
%18
	\prob{	Given is a convex quadrilateral $ABCD$ with $AB=CD$. Draw the triangles $ABE$ and $CDF$ outside $ABCD$ so that $\angle{ABE} = \angle{DCF}$ and $\angle{BAE}=\angle{FDC}$. Prove that the midpoints of $\overline{AD}$, $\overline{BC}$ and $\overline{EF}$ are collinear	}
	
%19
	\prob{	Let $P$ be a point out of circle $C$. Let $PA$ and $PB$ be the tangents to the circle drawn from $C$. Choose a point $K$ on $AB$ . Suppose that the circumcircle of triangle $PBK$ intersects $C$ again at $T$. Let ${P}'$ be the reflection of $P$ with respect to $A$. Prove that
		\[ \angle PBT = \angle {P}'KA \]	}
	
%20
	\prob{	Consider a circle $C_1$ and a point $O$ on it. Circle $C_2$ with center $O$, intersects $C_1$ in two points $P$ and $Q$. $C_3$ is a circle which is externally tangent to $C_2$ at $R$ and internally tangent to $C_1$ at $S$ and suppose that $RS$ passes through $Q$. Suppose $X$ and $Y$ are second intersection points of $PR$ and $OR$ with $C_1$. Prove that $QX$ is parallel with $SY$.	}
	
%21
	\prob{	In triangle $ABC$ we have $\angle A=\frac{\pi}{3}$. Construct $E$ and $F$ on continue of $AB$ and $AC$ respectively such that $BE=CF=BC$. Suppose that $EF$ meets circumcircle of $\triangle ACE$ in $K$. ($K\not \equiv E$). Prove that $K$ is on the bisector of $\angle A$	}
	
%22
	\prob{	In triangle $ABC$, $\angle A=90^{\circ}$ and $M$ is the midpoint of $BC$. Point $D$ is chosen on segment $AC$ such that $AM=AD$ and $P$ is the second meet point of the circumcircles of triangles $\Delta AMC,\Delta BDC$. Prove that the line $CP$ bisects $\angle ACB$	}
	
%23
	\prob{	Let $C_1,C_2$ be two circles such that the center of $C_1$ is on the circumference of $C_2$. Let $C_1,C_2$ intersect each other at points $M,N$. Let $A,B$ be two points on the circumference of $C_1$ such that $AB$ is the diameter of it. Let lines $AM,BN$ meet $C_2$ for the second time at $A',B'$, respectively. Prove that $A'B'=r_1$ where $r_1$ is the radius of $C_1$.	}
	
%24
	\prob{	Given a triangle $ABC$, let $P$ lie on the circumcircle of the triangle and be the midpoint of the arc $BC$ which does not contain $A$. Draw a straight line $l$ through $P$ so that $l$ is parallel to $AB$. Denote by $k$ the circle which passes through $B$, and is tangent to $l$ at the point $P$. Let $Q$ be the second point of intersection of $k$ and the line $AB$ (if there is no second point of intersection, choose $Q = B$). Prove that $AQ = AC$.	}
	
%25
	\prob{		Let $ABCD$ be a cyclic quadrilateral in which internal angle bisectors $\angle ABC$ and $\angle ADC$ intersect on diagonal $AC$. Let $M$ be the midpoint of $AC$. Line parallel to $BC$ which passes through $D$ cuts $BM$ at $E$ and circle $ABCD$ in $F$ ($F \neq D$ ). Prove that $BCEF$ is parallelogram	}
	
%26
	\prob{	The side $BC$ of the triangle $ABC$ is extended beyond $C$ to $D$ so that $CD = BC$. The side $CA$ is extended beyond $A$ to $E$ so that $AE = 2CA$. Prove that, if $AD=BE$, then the triangle $ABC$ is right-angled	}
	
%27
	\prob{	$ABCD$ is a cyclic quadrilateral inscribed in the circle $\Gamma$ with $AB$ as diameter. Let $E$ be the intersection of the diagonals $AC$ and $BD$. The tangents to $\Gamma$ at the points $C,D$ meet at $P$. Prove that $PC=PE$	}
	
%28
	\prob{	The quadrilateral $ABCD$ is inscribed in a circle. The point $P$ lies in the interior of $ABCD$, and $\angle P AB = \angle P BC = \angle P CD = \angle P DA$. The lines $AD$ and $BC$ meet at $Q$, and the lines $AB$ and $CD$ meet at $R$. Prove that the lines $P Q$ and $P R$ form the same angle as the diagonals of $ABCD$	}
	
%29
	\prob{	Let $ABCD$ be a cyclic quadrilateral with opposite sides not parallel. Let $X$ and $Y$ be the intersections of $AB,CD$ and $AD,BC$ respectively. Let the angle bisector of $\angle AXD$ intersect $AD,BC$ at $E,F$ respectively, and let the angle bisectors of $\angle AYB$ intersect $AB,CD$ at $G,H$ respectively. Prove that $EFGH$ is a parallelogram.	}
	
%30
	\prob{	Triangle $ABC$ is given with its centroid $G$ and cicumcentre $O$ is such that $GO$ is perpendicular to $AG$. Let $A'$ be the second intersection of $AG$ with circumcircle of triangle $ABC$. Let $D$ be the intersection of lines  $CA'$ and $AB$ and $E$ the intersection of lines $BA'$ and $AC$. Prove that the circumcentre of triangle $ADE$ is on the circumcircle of triangle $ABC$	}
	
%31
	\prob{	Let $M$ be the midpoint of the side $AC$ of $ \triangle ABC$. Let $P\in AM$ and $Q\in CM$ be such that $PQ=\frac{AC}{2}$. Let $(ABQ)$ intersect with $BC$ at $X\not= B$ and $(BCP)$ intersect with $BA$ at $Y\not= B$. Prove that the quadrilateral $BXMY$ is cyclic.	}
	
%32
	\prob{	Let be given a triangle $ ABC$ and its internal angle bisector $ BD$ $ (D\in BC)$. The line $ BD$ intersects the circumcircle $ \Omega$ of triangle $ ABC$ at $ B$ and $ E$. Circle $ \omega$ with diameter $ DE$ cuts $ \Omega$ again at $ F$. Prove that $ BF$ is the symmedian line of triangle $ ABC$.	}
	
%33
	\prob{	$ \Delta ABC$ is a triangle such that $ AB \neq AC$. The incircle of $ \Delta ABC$ touches $ BC, CA, AB$ at $ D, E, F$ respectively. $ H$ is a point on the segment $ EF$ such that $ DH \bot EF$. Suppose $ AH \bot BC$, prove that $ H$ is the orthocenter of $ \Delta ABC$.	}
	
%34
	\prob{Let $ ABC $ be a triangle and let $ P $ be a point on the angle bisector $ AD $, with $ D $ on $ BC $. Let $ E, F $ and $ G $ be the intersections of $ AP, BP $ and $ CP $ with the circumcircle of the
triangle, respectively. Let $ H $ be the intersection of $ EF $ and $ AC $, and let $ I $ be the intersection of
$ EG $ and $ AB $. Determine the geometric place of the intersection of $ BH $ and $ CI $ when $ P $ varies}
	
%35
	\prob{Let $ D; E; F $ be the points on the sides $ BC; CA; AB $ respectively, of $ \triangle ABC $. Let
$ P; Q; R $ be the second intersection of $ AD; BE; CF $ respectively, with the cricumcircle of $ \triangle ABC $.
Show that
	\[\frac{AD}{PD}+\frac{BE}{QE}+\frac{CF}{RF}\geq 9\]}

%36
	\prob{Points $ D $ and $ E $ lie on sides $ AB $ and $ AC $ of triangle $ ABC $ such that $ DE \parallel BC $.
Let $ P $ be an arbitrary point inside $ ABC $. The lines $ PB $ and $ PC $ intersect $ DE $ at $ F $ and $ G $,
respectively. If $ O_1 $ is the circumcenter of $ PDG $ and $ O_2 $ is the circumcenter of $ PFE $, show that
$ AP \parallel O_1O_2 $.}
	
%37
	\prob{Let $ ABC $ be a triangle. A circle passing through $ A $ and $ B $ intersects segments
$ AC $ and $ BC $ at $ D $ and $ E $, respectively. Lines $ AB $ and $ DE $ intersect at $ F $, while lines $ BD $ and
$ CF $ intersect at $ M $. Prove that $ MF = MC $ if and only if $ MB \cdot MD = MC^2 $}
	
%38
	\prob{Let $ O $ and $ I $ be the circumcenter and incenter of triangle $ ABC $, respectively. Let $ \omega A $ be the excircle of triangle $ ABC $ opposite to $ A $; let it be tangent to $ AB, AC, BC $ at
$ K, M, N $, respectively. Assume that the midpoint of segment $ KM $ lies on the circumcircle of triangle $ ABC $. Prove that $ O; N; I $ are collinear.}
	
%39
	\prob{ Let $ ABCD $ be a cyclic quadrilateral. Let $ AB \cap CD = P $ and $ AD \cap BC = Q $.
Let the tangents from $ Q $ meet the circumcircle of $ ABCD $ at $ E $ and $ F $. Prove that $ P; E; F $ are
collinear.}
	
	
\end{document}



