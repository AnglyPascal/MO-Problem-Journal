\documentclass[12pt]{article}

\RequirePackage{mathtools, amsmath, enumitem, amsfonts, fancyhdr, float, chngcntr}
\usepackage[utf8]{inputenc}
\RequirePackage[dvipsnames]{xcolor}
\usepackage{amsthm, amssymb}
\usepackage{algorithmic}

%\usepackage[draft]{graphicx}
\usepackage{graphicx}

\usepackage{thmtools, chngcntr, listings, titlesec, tocloft, setspace, multicol, caption, pdfpages, pgf, tikz, pgfplots, subfig, xr}

\declaretheoremstyle[
	headfont=\bfseries,
	headpunct={.},
	postheadspace={7pt},
	spacebelow=20pt,
]{problemstyle}

\declaretheorem[style=problemstyle]{problem}

\title{\vspace{-3em}National Math Camp 2020 Exam 2}

\begin{document}
\maketitle
\thispagestyle{empty}
\vspace{2em}

\begin{problem}
    Two circles $\Gamma_{1}$ and $\Gamma_{2}$ meet at $A$ and $B$. Let $r$ be
    a line through $B$ that meets $\Gamma_{1}$ at $C$ and $\Gamma_{2}$ at $D$,
    such that $B$ is between $C$ and $D$. Let $s$ be the line parallel to $A
    D$, which is tangent to $\Gamma_{1}$ in $E$ and has the minimal distance
    from $A D . E A$ meets $\Gamma_{2}$ in $F$, and let $t$ be the line
    through $F$ which is tangent to $\Gamma_{2}$. Prove that:\\

    a) $t \parallel A C$.\\

    b) $r, s$ and $t$ are concurrent.
\end{problem}

\begin{problem}
    Find all functions $f:\mathbb{R}\to\mathbb{R}$ that satisfy the
    following equation:
    \[f(xy) = \begin{cases}
        f(x)f(y) & \text{if } f(x+y) \le f(x)f(y)\\
        f(x+y) & \text{otherwise}
    \end{cases}\] 
\end{problem}

\begin{problem}
    Let $n$ be a positive integer. Prove that there exist integers $b_{1},
    b_{2}, \ldots, b_{n}$ such that for any integer $m,$ the number
    \[\left(\cdots\left(\left(\left(m^{2}+b_{1}\right)^{2}+b_{2}\right)^{2}+
    \cdots\right)^{2}+b_{n-1}\right)^{2}+b_{n}\]
    is divisible by $2 n-1$
\end{problem}

\begin{problem}
    $2015$ positive integers are arranged on a circle. The
    difference between any two adjacent numbers equals their greatest common
    divisor. Determine the maximal value of $N$ which divides the
    product of all $2015$ numbers, regardless of their choice.
\end{problem}
    
\end{document}
