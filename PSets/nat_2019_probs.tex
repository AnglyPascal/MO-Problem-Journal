\section{National Camp 2019 Problem Set}
	
	\prob{https://math.stackexchange.com/questions/675646/if-there-are-5-points-on-a-sphere-then-4-of-them-belong-to-a-half-sphere}{		}{E}{Given any five points on a sphere, show that some four of them must lie on a closed hemisphere.}
	
	
	\prob{}{}{}{Alice and Bob play a game. There is a threshold $ n $, Initially, the game starts with the number $ 1 $. In a move the player replaced the number $ m $ with either $ m+1 $ or $ 2m $. Assuming optimal play, count the number of threshold under $ 2019 $ where Alice wins.}
	
	
	\prob{}{}{}{For an integer $ n $, let $ a_1, a_2, \dots a_{\phi(n)} $ be the integers less than and coprime to $ n $. Determine all possible values of \[\prod_{i=1}^{\phi(n)} a_i\ (\text{mod}\ n)\] and for which values of $ n $ they appear.}
	
	
	\prob{}{}{}{Let $ O $ and $ N $ be the circumcenter and nine-point-center of $ \triangle ABC $ respectively. Let $ I_b $ and $ I_c $ be the $ B $ and $ C $ excenters of $ \triangle ABC $ respectively. Prove that \[ \angle I_bOI_c = 180^\circ - \frac 1 2 \angle I_bNI_c \]}
	
	
	\prob{https://artofproblemsolving.com/community/q3h448003p2521640}{Putnam 2001 A5}{E}{Prove that there are unique positive integers $ a, n $ such that \[a^{n+1} - (a+1)^n = 2001\]}
	
	
	\prob{}{}{}{Find all polynomials $ P(x) $ with real coefficients such that for all real numbers $ x+ y+ z= 0 $, then the determinant of the following matrix is $ 0 $ 
	\[
	\begin{bmatrix}
		1 & x & P(x)\\
		1 & y & P(y)\\
		1 & z & P(z)
	\end{bmatrix}
	\]
	}	

	
	
	
	\prob{https://artofproblemsolving.com/community/c6h60830p366681}{IMO 1971 P5}{M}{Prove that for every positive integer $m$ we can find a finite set $S$ of points in the plane, such that given any point $A$ of $S$, there are exactly $m$ points in $S$ at unit distance from $A$.}
	
	
	\prob{https://artofproblemsolving.com/community/q2h368177p2026501}{ISL 1971}{M}{Consider a sequence of polynomials $P_0(x), P_1(x), P_2(x), \ldots, P_n(x), \ldots$, where $P_0(x) = 2, P_1(x) = x$ and for every $n \geq 1$ the following equality holds:
		\[P_{n+1}(x) + P_{n-1}(x) = xP_n(x).\]
		Prove that there exist three real numbers $a, b, c$ such that for all $n \geq 1,$
		\[(x^2 - 4)[P_n^2(x) - 4] = [aP_{n+1}(x) + bP_n(x) + cP_{n-1}(x)]^2.\]}
	
	
	\prob{https://artofproblemsolving.com/community/q2h58616p359102}{IMO 1974 P3}{}{Prove that for any $ n $ natural, the number \[\sum_{k=0}^{n} \binom{2n+1}{2k+1}2^3k\] cannot be divided by $ 5 $.}
	
	
	\prob{https://artofproblemsolving.com/community/c7h513439p2882561}{Putnam 1999 B6}{M}{Let $S$ be a finite set of integers, each greater than $1$. Suppose that for each integer $n$ there is some $s\in S$ such that $\gcd(s,n)=1$ or $\gcd(s,n)=s$. Show that there exist $s,t\in S$ such that $\gcd(s,t)$ is prime.}
	
	
	\prob{https://artofproblemsolving.com/community/c6h279890p1511554}{Sharygin 2009 P4}{E}{Let $ P$ and $ Q$ be the common points of two circles. The ray with origin $ Q$ reflects from the first circle in points $ A_1$, $ A_2$,$ \ldots$ according to the rule ``the angle of incidence is equal to the angle of reflection''. Another ray with origin $ Q$ reflects from the second circle in the points $ B_1$, $ B_2$,$ \ldots$ in the same manner. Points $ A_1$, $ B_1$ and $ P$ occurred to be collinear. Prove that all lines $ A_iB_i$ pass through $ P $.}
	
	
	\prob{https://artofproblemsolving.com/community/q2h127832p725339}{Greece}{M}{Find all surjective functions $ f:\N \to \N $ such that for all $ m, n\in \N $\[m|n \Longleftrightarrow f(m)|f(n)\]}
	
	
	
	
	
	\prob{https://artofproblemsolving.com/community/c6h358783p1960094}{USA TST 2010 P9}{H}{Determine whether or not there exists a positive integer $k$ such that $p = 6k+1$ is a prime and
		\[\binom{3k}{k} \equiv 1 \pmod{p}\]}
	
	
	\prob{https://artofproblemsolving.com/community/c6h420428p2374811}{USA TST 2011 P6}{H}{A polynomial $P(x)$ is called nice if $P(0) = 1$ and the nonzero coefficients of $P(x)$ alternate between $1$ and $-1$ when written in order. Suppose that $P(x)$ is nice, and let $m$ and $n$ be two relatively prime positive integers. Show that
		\[Q(x) = P(x^n) \cdot \frac{(x^{mn} - 1)(x-1)}{(x^m-1)(x^n-1)}\]
		is nice as well.}
	
	
	\prob{}{}{}{Find all pairs of positive integers $ (m,n)$ such that $ mn - 1$ divides $ (n^2 - n + 1)^2$.}
	
	
	\prob{}{}{}{In isosceles $\triangle ABC$, $AB=AC$, points $D,E,F$ lie on segments $BC,AC,AB$ such that $DE\parallel AB$, $DF\parallel AC$. The circumcircle of $\triangle ABC$ $\omega_1$ and the circumcircle of $\triangle AEF$ $\omega_2$ intersect at $A,G$. Let $DE$ meet $\omega_2$ at $K\neq E$. Points $L,M$ lie on $\omega_1,\omega_2$ respectively such that $LG\perp KG, MG\perp CG$. Let $P,Q$ be the circumcenters of $\triangle DGL$ and $\triangle DGM$ respectively. Prove that $A,G,P,Q$ are concyclic.}
	
		\figdf{.3}{nat_pset_16}{}
	
	
	\prob{}{}{}{Given a polynomial $f(x)$ with rational coefficients, of degree $d \ge 2$, we define the sequence of sets $f^0(\mathbb{Q}), f^1(\mathbb{Q}), \ldots$ as $f^0(\mathbb{Q})=\mathbb{Q}$, $f^{n+1}(\mathbb{Q})=f(f^{n}(\mathbb{Q}))$ for $n\ge 0$. (Given a set $S$, we write $f(S)$ for the set $\{f(x)\mid x\in S\})$.
		Let $f^{\omega}(\mathbb{Q})=\bigcap_{n=0}^{\infty} f^n(\mathbb{Q})$ be the set of numbers that are in all of the sets $f^n(\mathbb{Q})$, $n\geq 0$. Prove that $f^{\omega}(\mathbb{Q})$ is a finite set.}
	
	
	\prob{}{}{}{Define the polymonial sequence $\left \{ f_n\left ( x \right ) \right \}_{n\ge 1}$ with $f_1\left ( x \right )=1$, $$f_{2n}\left ( x \right )=xf_n\left ( x \right ), \; f_{2n+1}\left ( x \right ) = f_n\left ( x \right )+ f_{n+1} \left ( x \right ), \; n\ge 1.$$Look for all the rational number $a$ which is a root of certain $f_n\left ( x \right ).$}
	
	
	
	
	
	\prob{https://artofproblemsolving.com/community/q1h547874p3176078}{Sharygin 2013 Final Round 10.8}{}{Two fixed circles are given on the plane, one of them lies inside the other one. From a point $ C $ moving arbitrarily on the external circle, draw two chords $ CA, CB $ of the larger circle such that they tangent to the smaller one. Find the locus of the incenter of triangle $ ABC $.}
	
	
	\prob{https://artofproblemsolving.com/community/q1h1279375p6724144}{IMC 2014 P3}{}{Let $n$ be a positive integer. Show that there are positive real numbers $a_0, a_1, \dots, a_n$ such that for each choice of signs the polynomial
		\[\pm a_nx^n\pm a_{n-1}x^{n-1} \pm \dots \pm a_1x \pm a_0\]
	has $n$ distinct real roots.}
	
	
	\prob{https://artofproblemsolving.com/community/c260h1442608p8732088}{Dunno}{}{Suppose that the real numbers $a_0,a_1,...,a_n$ and $x$, with $0<x<1$, satisfy
		\[\frac{a_0}{1-x}+\frac{a_1}{1-x^2}+...+\frac{a_n}{1-x^{n+1}}=0\]	
	Prove that there exist a real number $y$ with $0<y<1$ such that
		\[a_0+a_1y+a_2y^2+...+a_ny^n=0\]}
	
	
	
	\prob{https://artofproblemsolving.com/community/q2h523187p2952670}{RMM 2013 P6}{}{A token is placed at each vertex of a regular $2n$-gon. A move consists in choosing an edge of the $2n$-gon and swapping the two tokens placed at the endpoints of that edge. After a finite number of moves have been performed, it turns out that every two tokens have been swapped exactly once. Prove that some edge has never been chosen.}
	
	
	
	\prob{https://artofproblemsolving.com/community/c6h1683076p10734638}{Sharygin 2018 Final Round 8.4}{}{Find all sets of six points in the plane, no three collinear, such that if we partition the set into two sets, then the obtained triangles are congruent.}


