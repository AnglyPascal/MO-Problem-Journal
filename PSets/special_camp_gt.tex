\thispagestyle{empty}
\begin{center}
	\Huge Special Online Math Camp 2020 \\[1em]
	\LARGE Graph Theory Problem Set \\[1em]
	\normalsize \signature\\
\end{center}


The problems are roughly in order of difficulty. Sources were omitted to hide the exact difficult of the problems. Try them all. Many of these problems require representing the problem in terms of graphs in a clever way. But they can all be solved without any heavy techniques. So... Best of luck.

\begin{enumerate}
	\item Given a bipartite graph, prove that the minimum number of colors required to color the edges of the graph such that no node is adjacent to $ 2 $ edges of same color is the maximum degree of the graph.
	
	\item In a country some cities are connected by oneway flights (There are no more then one flight between two cities). City $ A $ called "available" for city $ B $ , if there is flight from $ B $ to $ A $ , maybe with some transfers. It is known, that for every 2 cities $ P $ and $ Q $ exist city $ R $ , such that $ P $ and $ Q $ are available from $ R $. Prove, that exist city $ A $ , such that every city is available for $ A $.

	\item $ 100 $ people from $ 50 $ countries, two from each countries, stay on a circle. Prove that one may partition them onto $ 2 $ groups in such way that neither no two countrymen, nor three consecutive people on a circle, are in the same group.
	
	\item There are $ 100 $ people from $ 25 $ countries sitting around a circular table. Prove that they can be separated into four classes, so that no two countrymen are in the same class, nor any two people sitting adjacent in the circle.
	
	\item A house has an even number of lamps distributed among its rooms in such a way that there are at least three lamps in every room. Each lamp shares a switch with exactly one other lamp, not necessarily from the same room. Each change in the switch shared by two lamps changes their states simultaneously. Prove that for every initial state of the lamps there exists a sequence of changes in some of the switches at the end of which each room contains lamps which are on as well as lamps which are off.
	
	\item Call a graph \textit{symmetric}, if one can put its vertices on the plane such that it becomes symmetric wrt a line (which doesn't pass through any vertex). Find the minimum value of $ k $ such that (the edges of) every graph on $ 100 $ vertices, can be decomposed into $ k $ symmetric subgraph.
	
	\item We have a closed path that goes from one vertex to another neighboring vertex, on the vertices of a $ n\times n$ square which pass throgugh each vertex exactly once. Prove that we have two adjacent vertices such that if we cut the path at these two points then the length of each open paths is at least $ n^2/4 $.
	
	\item A crazy physicist discovered a new kind of particle which he called an $ i $ -mon, after some of them mysteriously appeared in his lab. Some pairs of $ i $ -mons in the lab can be entangled, and each $ i $ -mon can participate in many entanglement relations. The physicist has found a way to perform the following two kinds of operations with these particles, one operation at a time.
	
	\begin{enumerate}
		
		\item If some $ i $ -mon is entangled with an odd number of other $ i $ -mons in the lab, then the physicist can destroy it.
		
		\item At any moment, he may double the whole family of $ i $ -mons in the lab by creating a copy $ I' $ of each $ i $ -mon $ I $. During this procedure, the two copies $ I' $ and $ J' $ become entangled if and only if the original $ i $ -mons $ I $ and $ J $ are entangled, and each copy $ I' $ becomes entangled with its original $ i $ -mon $ I $ ; no other entanglements occur or disappear at this moment.
	\end{enumerate}

	Prove that the physicist may apply a sequence of much operations resulting in a family of $ i $ -mons, no two of which are entangled.

	\item A communications network consisting of some terminals is called a $3$-connector if among any three terminals, some two of them can directly communicate with each other. A communications network contains a windmill with $n$ blades if there exist $n$ pairs of terminals $\{x_{1},y_{1}\},\{x_{2},y_{2}\},\ldots,\{x_{n},y_{n}\}$ such that each $x_{i}$ can directly communicate with the corresponding $y_{i}$ and there is a hub terminal that can directly communicate with each of the $2n$ terminals $x_{1}, y_{1},\ldots,x_{n}, y_{n}$ . Determine the minimum value of $f (n)$, in terms of $n$, such that a $3$ -connector with $f (n)$ terminals always contains a windmill with $n$ blades.

	\item $ 2n $ real numbers with a positive sum are aligned in a circle. For each of the numbers, we can see there are two sets of $ n $ numbers such that this number is on the end. Prove that at least one of the numbers has a positive sum for both of these two sets.
\end{enumerate}