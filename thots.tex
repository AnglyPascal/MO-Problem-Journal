\chapter{Thoughts on PSolving, a note to thyself}
\thispagestyle{empty}
\section{Be DUMB, Keep it SIMPLE} 

Remember what Paul Zeitz said? Think wishfully, make dumb wishes. When first approaching the problem, you can do whatever you want. You can loosen some constraints, you can add some new. This works exceptionally well when you need to build an object from one given object, you can do whatever you want. Putting additional constraints decreases the number of test cases. Sometimes loosening some constraints help to give better observation of the problem. \\


Like in ISL 2016 N5, after deciding that we are going to build a pair $ (x_2, y_2) $ from the previous pair $ (x_1, y_1) $, we should look for the most innocent looking relation between these four variable. Now it's time to play around, try dumb things. Rewriting the equation, we want to use the fact that $ x_1, x_2 $ have to be on different sides of $ \sqrt{a} $. How can we insert this constraint in our equation in the most simple and natural way? This is where we need to be dumb, and amature.\\


In \autoref{problem:simurgh_2019_p3} the trick is to keep things simple. Making the most natural assumptions. In construction problems, think of how the result can be achieved in the most natural way. Can we make some extra assumptions that might result in the immediate proof the result's existence? 


\newpage
\section*{What to do in a contest}

\begin{enumerate}
    \item Avoid immediate fixation on one approach.
    \item Try to be wishful. Try some smaller cases first. But don't spend too
        much time behind finding patterns. \textbf{Keep a balance between
        working out examples and conjecturing outcomes.}
    \item During investigating, focusing on only approach won't do good.
        During the investigation phase, try out all the approaches at once,
        and then go with the most promising ones. But mass-create examples and
        approaches at first.
    \item \textbf{First attacks:}
        \begin{enumerate}
            \item First study the problem carefully. Don't miss any details.
                That means in a geo problem, naming the points and writing the
                corrent condition and conclusions, and in combi/alg/nt
                problems, checking positive vs negative or finite vs infinite
                conditions.
            \item Try to classify the problems with previous encounters. Maybe
                remember some popular ways to approach it.
            \item Which arguments seem the most plausible? What kind of
                solution there might be? Will it use some construction? Maybe
                we will need to show contradiction assuming otherwise, or we
                might need to use projective geo?
            \item Repeat at least twice. 
            \item Specially for geo and graphs, try to find an easier
                construction? Maybe we can think of graphs instead of grids,
                or maybe we can interpret in another way?
        \end{enumerate}
    \item Now examine the problem closely:
        \begin{enumerate}
            \item \textbf{Get your hands dirty}: write down possible solutions
                for functional equations, or workout some smaller cases.
            \item \textbf{Penultimate Steps}: Try to think of the situations,
                where the conclusion would follow trivially from. Write them
                down.
            \item \textbf{Wishful thinking, Making it easier}: Try to simplify
                the problems using stricter assumtions, or maybe loosing some
                constraints. Play out with the conditions.
        \end{enumerate}
\end{enumerate}
