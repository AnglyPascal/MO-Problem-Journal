\section{NT Functions and Polynomials}

\begin{Remark}
    The main idea for most NT FEs is to take aid from LARGE integers on the
    left side of the divisibility.
\end{Remark}

\prob{https://artofproblemsolving.com/community/c6h597243p3544096}{ISL 2013 N1}{E}{Let $\mathbb{Z} _{>0}$ be the set of positive integers. Find all functions $f: \mathbb{Z} _{>0}\rightarrow \mathbb{Z} _{>0}$ such that
    \[ m^2 + f(n) \mid mf(m) +n \]
for all positive integers $m$ and $n$.}

\solu{Go with the flow.}



\prob{https://artofproblemsolving.com/community/c6h356076p1935854}{ISL 2010 N5}{M}{Find all functions $g:\mathbb{N}\rightarrow\mathbb{N}$ such that \[\left(g(m)+n\right)\left(g(n)+m\right)\] is a perfect square for all $m,n\in\mathbb{N}.$}

\solu{Playig around with some primes give us that for every ``big'' primes, we need to have every residue class present in the range of $ g $. Now with this fact, we can prove the injectivity as well. Now we want to show that $ g(n+1)=g(n)+1 $. How to show that? We can show that by saying that no prime $ p $ exits such that $ p|g(n+1)-g(n) $, in other words, again the residue classes.}



\prob{https://artofproblemsolving.com/community/c6h27188p169519}{ISL 2004 N3}{E}{Find all functions $ f: \N\to\N$ satisfying
    \[ \left. \left(f\left(m\right)^{2}+f\left(n\right)\right) \right|
        \left(m^{2}+n\right)^{2}\]
for any two positive integers $ m$ and $ n$.}

\begin{solution}
    First get $ f(1)=1 $, then using casework, get $ f(p-1)=p-1 $. Then let $
    m=(p-1) $, and use the division algorithm to reduce the dividend.
\end{solution}

\begin{Remark}
    In nt-fe s with ``divisible'' condition, one occurring idea is to get an
    infinite set of integers with the desired output value, and somehow keep
    those numbers only in the divisor, and make them vanish from the dividend.
    Most of the time, using division algorithm to reduce the dividend does the
    trick.
\end{Remark}



\prob{https://artofproblemsolving.com/community/c6h355799p1932945}{ISL 2009 N5}{E}{Let $P(x)$ be a non-constant polynomial with integer coefficients. Prove that there is no function $T$ from the set of integers into the set of integers such that the number of integers $x$ with $T^n(x)=x$ is equal to $P(n)$ for every $n\geq 1$, where $T^n$ denotes the $n$-fold application of $T$.}

\solu{Write down what the problem is actually saying. Integer functions are like cycles, use that.}


\prob{}
{ISL 2019 N4}{M}{
    Find all functions $f:\mathbb{N} \to \mathbb{N}$ such that for all $a, b
    \in \mathbb{N}, a+b < C$ for some constant $C$, we have
    \[a + f(b) | a^2 + bf(a)\] 
}

\begin{solution}
    The main idea is to get primes on the left hand side somehow. We want to
    show for primes, $p|f(p)$.
\end{solution}


