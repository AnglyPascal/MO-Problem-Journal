% !TEX program = xelatex
\documentclass[compress]{beamer}

\usepackage[english]{babel}
\usepackage{metalogo}
\usepackage{listings}
\usepackage{fontspec}
\usepackage{graphicx}

\usetheme{NordWhite}

\setsansfont{Andika New Basic}
\setmainfont{Yanone Kaffeesatz}
\setmonofont{Share Tech Mono}

\AtBeginSection[] {
    \begin{frame}[c,noframenumbering,plain]
        \tableofcontents[sectionstyle=show/hide,subsectionstyle=show/show/hide]
    \end{frame}
}

\AtBeginSubsection[] {
    \begin{frame}[c, noframenumbering,plain]
        \tableofcontents[sectionstyle=show/hide,subsectionstyle=show/shaded/hide]
    \end{frame}
}

\newcommand{\gap}{\pause\vspace{1em}}

\title{Orders Modulo A Prime}
\subtitle{How to make life easier}
\author{M Ahsan Al Mahir}
\date{\today}

\begin{document}
\begin{frame}[plain,noframenumbering]
    \maketitle
\end{frame}

\section{Recap}

\subsection{Euler and Fermat's Theorem}

\begin{frame}
    We all know the following theorem, right?
    \[a^{\phi(n)} \equiv 1 \ (\text{mod } n)\]
    Where $\phi(n)$ is the number of numbers that are coprime to $n$ and are
    smaller than $n$. 

    \gap

    We can use this theorem in the following way:
    \[a^m \equiv a^{m \ (\text{mod } \phi(n))} \ (\text{mod } n)\] 

\end{frame}

\begin{frame}
    So for example, if $n = 12, \phi(12) = 4$ and so:
    \[7^{13} \equiv 7^{13 \ (\text{mod } 4)} \equiv 7^1 \ (\text{mod } 12)\] 
\end{frame}

\subsection{Modular Inverses}

\begin{frame}
    If $a$ and $n$ are \texttt{coprime integers}, then there exists a positive integer
    $b<n$ such that 
    \[ab \equiv 1 \ (\text{mod } n)\]

    \gap 

    Why do we need inverses? \pause We can use them in place of division by
    $a$:

    \[\frac{1}{a} \equiv b \ (\text{mod } n)\]

    \gap

    So adding or multiplying by $\frac{1}{a}$ is just adding or multiplying by
    $b$ modulo $n$.
\end{frame}

\begin{frame}
    So if there are two integers $s, t$ such that 
    \[a^s \equiv a^t \ (\text{mod } n)\]
    then we can say
    \[a^{s-t} \equiv 1 \ (\text{mod } n)\]

    \gap

    \textcolor{NordRed}{In other words, if $a$ is coprime to $n$, we can
    divide congruences modulo $n$ by $a$.}
\end{frame}

\begin{frame}
    Note that if $\gcd\left(a, n\right)\neq 1$, then it doesn't hold. 

    \gap

    $6$ can never divide $2a - 1$. 

    \gap
    Just like that $\gcd(n, a)$ must divide $1$, because it divides $ab-1$.
    And so the gcd has to be $1$.
\end{frame}

\subsection{Bezout's Identity}

\begin{frame}
    Another useful identity to remember is \textcolor{NordOrange}{for all
        integer $a, b$, we will
        find two integers $x, y$ such that 
        \[ax + by = \gcd\left(a, b\right) \]
    } 
\end{frame}


\section{Orders}
\subsection{Why do we care?}

\begin{frame}
    \textcolor{NordOrange}{
        Suppose you are told to count the remainder of $10^{561}$ when divided by
        $73$, how would you do it?
    }

    \gap

    One trick you learned in \textbf{Fermat's Little Theorem} is to reduce the
    exponent by $\phi(73)$ which happens to be $72$.

    \gap

    You can do:
    \begin{align*}
        10^{72} \equiv 1\ (\text{mod } 73)\\
        \text{ and, } 561 \equiv 57\ (\text{mod } 72)\\
        \text{so, } 10^{561} \equiv 10^{57} \ (\text{mod } 73) 
    \end{align*}

    \gap
    But then again, you need to count \textcolor{NordRed}{$10^{57}$} which is
    near impossible!!
\end{frame}

\begin{frame}
    But what if I told you,
    \textcolor{NordRed}{\[10^8  \equiv 1 \ (\text{mod } 73)\ \ ?\]}

    \gap

    Then your life would become so easy that you could just write
    \[10^{561} \equiv 10^{561 \ (\text{mod } 8)} \equiv 10^1 \ (\text{mod } 73)\] 
\end{frame}

\subsection{So what is ``Order''?}

\begin{frame}{Definition}
    \textcolor{NordBrightBlue}{If $a$ and $n$ are coprime integers, then the
        \textcolor{NordRed}{Order of $a$ modulo $n$} is the smallest integer $m$ such
        that 
        \textcolor{NordRed}{\[a^m \equiv 1 \ (\text{mod } n)\]}
    }

    \pause

    We write it as \textcolor{NordRed}{\[\text{Ord}_n(a) = m\]}

    \pause

    In our previous example, since $10^8 \equiv 1 \ (\text{mod } 73)$, we say
    \[\text{Ord}_{73}(10) = 8\] 
\end{frame}

\begin{frame}
    As more example, below are given the orders of $a$ modulo
    $11$ and $13$:
    \[
        \begin{array}{r|rr}
            a & \text{mod $11$} & \text{mod $13$} \\ \hline
            1  &  1 &  1 \\
            2  & 10 & 12 \\
            3  &  5 &  3 \\
            4  &  5 &  6 \\
            5  &  5 &  4 \\
            6  & 10 & 12
        \end{array}
        \qquad
        \begin{array}{r|rr}
            a & \text{mod $11$} & \text{mod $13$} \\ \hline
            7  & 10 & 12 \\
            8  & 10 &  4 \\
            9  &  5 &  3 \\
            10 &  2 &  6 \\
            11 &    & 12 \\
            12 &    &  2
        \end{array}
    \]

\end{frame}

\subsection{Important Properties of Order}

\begin{frame}
    An interesting thing you might have noticed in the last slide: all of the
    orders modulo $11$ were factors of $10 = \phi(11)$, that are $1, 2, 5,
    10$.

    \gap

    Is this is a coincident or are there more to it?
\end{frame}

\begin{frame}
    It turns out it is a fundamental property of orders:

    \gap

    \textcolor{NordRed}{If $m$ is the order of $a$ modulo $n$, then 
    $m | \phi(n)$}

    \gap

    How do you prove it? 

    \gap

    Suppose $m\not|\ \phi(n)$. So $\phi(n) = mq + r$ for
    some $q$ and $r$ with $r <m$. (Euclidean division)

    \gap

    Which gives us:

    \[a^{\phi(n)} \equiv a^{mq + r} \equiv {\left(a^m\right)}^q \times a^r \equiv a^r \equiv
    1 \ (\text{mod } n)\] 

    But $m$ is the smallest such positive number, so $r$ can't be smaller than $m$
    unless $r =0$ !
\end{frame}

\begin{frame}
    \textcolor{NordOrange}{
        In fact if $m = \text{Ord}_n(a)$ and for some $x$, 
        \[a^x \equiv 1 \ (\text{mod } n)\]
        then $m | x$
    }

    \vspace{1em}

    I leave its proof to you!
\end{frame}

\begin{frame}
    \textcolor{NordOrange}{If for some positive integer $s, t$ we have 
        \[a^s \equiv a^t \ (\text{mod } n)\] then $s \equiv t \ \left(\text{mod }
        \text{Ord}_n(a)\right)$ 
    }

    \gap

    We solve it by remembering that $a^{s-t} \equiv 1 \ (\text{mod } n)$ and
    that $\text{Ord}_n(a) | s-t$. So,
    \[s \equiv t \ (\text{mod } \text{Ord}_n(a))\] 

\end{frame}

\begin{frame}
    Now we use the idea of orders to prove the following very important
    theorem:

    \vspace{1em}

    \textcolor{NordRed}{For an odd prime $p$, if $a^2 \equiv -1 \ (\text{mod }
        p)$, then 
    \[4|p-1\]}

    \gap

    Can you solve it using the properties discussed earlier?
\end{frame}

\begin{frame}
    First we sqaure up the congruence to get
    \textcolor{NordRed}{\[\left(a^2\right) ^2 = a^4 \equiv 1 \ (\text{mod } p)\] }

    \pause

    \textcolor{NordBrightBlue}{Then we can say $\text{Ord}_p(a) | 4$ and so
    $\text{Ord}_p(a)$ is either $1, 2, 4$.} 

    \gap 

    But $\text{Ord}_p(a)$ can't be $1, 2$. Can you see why?

    \gap

    Because then $a^2$ would not be congruent to $-1$ modulo $p$. 

    \gap

    So \textcolor{NordRed}{$\text{Ord}_p(a) = 4$} and that gives us:
    \[4|\phi(p) \implies 4|p-1\] 
\end{frame}

\subsection{Usage in Problems}

\begin{frame}
    Now that we know what orders are, we can try using them in solving
    problems. Consider the following problem:

    \gap

    \textcolor{NordOrange}{Prove that for all integers $a, n$, we have $n|\phi(a^n-1)$.}

    \gap

    A lot of number theory problems need to be solved by cleverly finding
    which modulo to take. This is no exception.

    \gap

    From the idea of orders, we know that $\text{Ord}_m | \phi\left(m\right) $
    right?

    \gap

    \textcolor{NordBrightBlue}{
        So we want to make $n$ as the order of some integer modulo $a^n-1$.
    }

\end{frame}

\begin{frame}
    It turns out $a$ is the integer that we are looking for!! How?

    \gap

    We have $a^n \equiv 1 \ (\text{mod } a^n-1)$. Now for any integer $m < n$,
    we have 
    \[a^m -1 < a^n-1 \implies a^n-1\not| a^m-1\]

    \pause

    So $n$ is the smallest integer such that $a^n \equiv 1 \ (\text{mod }
    a^n-1)$, and so 
    \textcolor{NordRed}{\[\text{Ord}_{a^n-1}(a) = n\] }

    \gap

    And so we have 
\[n \left| \phi(a^n-1)\right.\] 
\end{frame}

\begin{frame}
    Another problem:

    \vspace{1em}

    \textcolor{NordOrange}{Prove that if $p$ is a prime number, then every
    prime divisor of $2^p-1$ is greater than $p$.}

    \gap

    A general tip for these kind of ``everything is greater than'' problem, it
    is usually helpful to assume the contrary. 

    \gap

    In our case, it will help if instead of \texttt{showing that all the prime
        factors are indeed greater than $p$}, \textcolor{NordBrightBlue}{we assume
    that there is a prime factor $q$ smaller than $p$}, and show contradiction.
\end{frame}

\begin{frame}
    So suppose there exists a prime $q<p$ such that 
    \[q|2^p-1 \implies 2^p \equiv 1 \ (\text{mod } q)\] 

    \gap

    We use our knowledge of orders and say that
    \textcolor{NordBrightBlue}{there is an integer $n\le q-1$ such that $2^n
    \equiv 1 \ (\text{mod } q)$}. Do you see the complication here?

    \gap

    Yes! If such $n$ existed, then \textcolor{NordRed}{that would mean $n|p$}!
    But it can't be true, since \textcolor{NordRed}{$n\le q-1 < p$ , but
    divides $p$}, a prime number. 

    \gap

    Wait, but $1|p$, so if $n=1$, it may happen right? 

    \gap

    No! That would mean \textcolor{NordRed}{$q|2^1-1$} which means $q|1$...
\end{frame}

\begin{frame}
    So there can't be a prime $q < p$ that divides $2^p-1$, and our problem is
    solved!
\end{frame}

\begin{frame}
    Another problem similar to the previous one:

    \vspace{1em}

    \textcolor{NordOrange}{Prove that if $p$ is a prime, then there is a prime
    greater than $p$ that divides $p^p-1$.}

    \gap

    Another prime exponent! Remember what we did just a while ago?

    \gap

    We take a prime smaller than $q$ that divides $p^p-1$. We will have that 
    \textcolor{NordRed}{\[\text{Ord}_q(p) | p\]}

    \pause

    And so $\text{Ord}_q(p) = 1$ and:
    \textcolor{NordRed}{\[q|p-1\]} 
\end{frame}

\begin{frame}
    Do you remember the factorization formula of $a^n-1$??
    \textcolor{NordBrightBlue}{ \[a^n-1 = (a-1)\left(a^{n-1} + a^{n-2} \dots +
    a + 1\right) \] }

    \pause

    So, 
    \[p^p-1 = \left(p-1\right) \left(p^{p-1} + p^{p-2} \dots  + p+1\right) \] 

    \pause

    We have found that if $q<p$, and $q|p^p-1$, then $q|p-1$.
    \textcolor{NordRed}{But what about the $\left(p^{p-1} + p^{p-2} \dots  +
    p+1\right)$ part?} No smaller prime divides this number. 

    \gap

    So there \textcolor{NordBrightBlue}{must be a prime bigger than $p$ that
    divides this number}!!
\end{frame}

\begin{frame}
    Now \textcolor{NordOrange}{prove that there is a prime factor of $p^p-1$
    of the form $pk+1$.}

    \gap

    Just before we proved that there is a prime $q$ larger than $p$ that
    divides $p^p-1$. Then by order's properties, we need to have
    \textcolor{NordRed}{\[\text{Ord}_q(p) = p \text{ or } \text{Ord}_q(p) = 1\]}

    \pause

    But it won't be $1$ ! \pause \textcolor{NordBrightBlue}{Because then
    $q|p-1$, but $q$ is bigger than $p$}!

    \gap

    So \textcolor{NordRed}{the order of $p$ modulo $q$ is $p$, and so
    $p|\phi(q)$}, which means $p|q-1$. 

    \gap That means \textcolor{NordBrightBlue}{$q-1 = pk \implies q=pk+1$ for
    some integer $k$}.

\end{frame}


\subsection{Something more fundamental}

\begin{frame}
    As you have seen, orders are really interesting when the modulo is a
    prime. And since the order always divide $p-1$, a natural question comes
    up:

    \gap

    \textcolor{NordRed}{When is the order equal to $p-1$?}

    \gap

    We have a special name for such $a$'s for which 
    \[\text{Ord}_p(a) = p-1\] 
\end{frame}

\begin{frame}{Primitive Roots}
    \textcolor{NordRed}{We call an integer $a$ smaller than $p$ a
    \texttt{Primitive Root} modulo $p$ if the order of $a$ modulo $p$ is $p-1$.}

    \gap

    In our very first example, we saw that for $11$, the primitive roots mod
    $11$ are 
    \[2, 6, 7, 8\]\pause
    And for $13$ the primitive roots mod $13$ are
    \[2, 6, 7, 11\] 

    \pause

    We won't dive too deep into the world of primitive roots, since it is
    really Huge, we will just see why they are so important!
\end{frame}

\begin{frame}
    \textcolor{NordOrange}{If $g$ is primitive root modulo $p$, then the two
        sets 
        \[\left\{g^1, g^2 \dots g^{p-1}\right\} \text{ and } \left\{1, 2,
        3\dots  p-1\right\}\] 
        are equal in modulo $p$.
    }

    \gap

    In the case of $11$, we can work it out by hand, if we take the primitive
    root $2$ the left set becomes:

    \textcolor{NordBrightBlue}{\[
            \begin{array}{r|r}
                g^k & \text{mod $11$} \\ \hline
                2 & 2 \\
                4 & 4  \\
                8 & 8  \\
                16 & 5 \\
                32 & 10  \\
            \end{array}\qquad
            \begin{array}{r|r}
                g^k & \text{mod $11$} \\ \hline
                64 & 9 \\
                128 & 7  \\
                265 & 3  \\
                512 & 6 \\
                1024 & 1  \\
            \end{array}
    \]}
\end{frame}

\begin{frame}
    So if we can find a primitive root, we can \textcolor{NordRed}{just keep
        multiplying it to itself and get the whole set $\left\{1, 2, \dots
    p-1\right\}$}. Pretty cool huh?
\end{frame}

\subsection{Conclusion}

\begin{frame}
    In short, orders are the smallest integer $m$ that gives us $a^m\equiv 1 \
    (\text{mod } n)$, and they appear everywhere once you start looking for
    them.
\end{frame}

\begin{frame}{Further Reading}
    \textcolor{NordOrange}{$a^n\pm 1$} by Yufei Zhao

    \gap

    \textcolor{NordOrange}{Orders Modulo A Prime} by Evan Chen

    \gap

    \textcolor{NordOrange}{Topics in Number Theory: An Olympiad-Oriented
    Approach} by Masum Billal and Amir Hossein Parvardi
\end{frame}

\end{document}
