\newpage\section{Length Relations}
	
	
	
		
		\lem{E.R.I.Q. (Equal Ration in Quadrilateral) Lemma}{Let $ A_1, B_1, C_1;\ A_2, B_2, C_2 $ be two sets of collinear points such that \[ \frac{A_1B_1}{B_1C_1} = \frac{A_2B_2}{B_2C_2} = k \]. Let points $ A, B, C $ be on $ A_1A_2, B_1B_2, C_1, C_2 $ such that: \[ \frac{A_1A}{A_2A} = \frac{B_1B}{B_2B} = \frac{C_1C}{C_2C} \] Then we have, \[ A, B, C \text{ are collinear and, } \frac{AB}{BC} = k  \]
			\fig{1}{ERIQ_Lemma}{E.R.I.Q. Lemma}
		}\label{eriq_lemma}
		
			
		
		\solu{A great use of this problem is in proving some midpoints collinear. Line in Newton-Gauss Line and some other such problems (\hrf{eriq_lemma_1}{1}, \hrf{eriq_lemma_2}{2}, \hrf{eriq_lemma_3}{3}) where it is asked to proved that some midpoints are collinear.}
		
		
		
		
		\lem{Steiner's Isogonal Cevian Lemma}{In $ \triangle ABC $, $ AA_1, AA_2 $ are two isogonal cevians, with $ A_1, A_2 \in BC $. Then we have \[ \frac{BA_1}{A_1C} \times \frac{BA_2}{A_2C} = \frac{BA^2}{AC^2} \]}
		
		
		
		
		\theo{}{}{Let $ P_1,P_2 $ be two isogonal conjugates wrt $ \triangle ABC $. Then if the Pedal triangle of $ P_1 $ is homological wrt to $ \triangle ABC $ then so is the Pedal triangle of $ P_2 .$}
		
		
		
		
		
		\theo{}{Erdos-Mordell Theorem (Forum Geometricorum Volume 1 (2001) 7-8)}{If from a point $O$ inside a given $\triangle ABC$ perpendiculars $OD,OE,OF$ are drawn to its sides, then $OA+OB+OC \geq 2(OD+OE+OF)$. Equality holds if and only if $\triangle ABC$ is equilateral.}
		
		Apparently nothing is needed except ``Ptolemy's Theorem''. Think of a way to connect $OA$ with $OE, OF$ and the sides of the triangle. As it is the most natural to use $AB, AC$, we have to deal with $BE, CF$ too. And dealing with lengths is the easiest when we have similar triangles. So we do some construction. 
		



		\prob{https://artofproblemsolving.com/community/c6h488832p2739339}{ISL 2011 G7}{E}{Let $ABCDEF$ be a convex hexagon all of whose sides are tangent to a circle $\omega$ with centre $O$. Suppose that the circumcircle of triangle $ACE$ is concentric with $\omega$. Let $J$ be the foot of the perpendicular from $B$ to $CD$. Suppose that the perpendicular from $B$ to $DF$ intersects the line $EO$ at a point $K$. Let $L$ be the foot of the perpendicular from $K$ to $DE$. Prove that $DJ=DL$.}
		
			\solu{There are a LOT of equal lenths, equal angles, and we have a perpendicularity lemma working as well. Why don't we try cosine :0}

	
