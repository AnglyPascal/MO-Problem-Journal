\newpage\section{Parallelogram Stuff}

	
	\theo{http://forumgeom.fau.edu/FG2005volume5/FG200510.pdf}{Maximality of the Area of a Cyclic Quadrilateral}{Among all quadrilaterals with given side lengths, the cyclic one has maximal area.
		\fig{.5}{quad-same_sides-diff_area}{The cyclic quad has the maximal area}
	}
		
		

	\prob{https://artofproblemsolving.com/community/c6h1508225_simple_parallelogram_geo}{IOM 2017 P1}{E}{Let $ABCD$ be a parallelogram in which angle at $B$ is obtuse and $AD>AB$. Points $K$ and $L$ on $AC$ such that $\angle ADL=\angle KBA$(the points $A, K, C, L$ are all different, with $K$ between $A$ and $L$). The line $BK$ intersects the circumcircle  $\omega$ of $ABC$ at points $B$ and $E$, and the line $EL$ intersects $\omega$ at points $E$ and $F$. Prove that $BF\parallel AC$.}

	\hl{Simplify}: Make the diagram easier to draw.


	\prob{https://artofproblemsolving.com/community/c6h148830p841255}{USA TST 2006 P6}{E}{Let $ABC$ be a triangle. Triangles $PAB$ and $QAC$ are constructed outside of triangle $ABC$ such that $AP = AB$ and $AQ = AC$ and $\angle{BAP}= \angle{CAQ}$. Segments $BQ$ and $CP$ meet at $R$. Let $O$ be the circumcenter of triangle $BCR$. Prove that $AO \perp PQ.$}
	
			\figdf{.5}{USATST2006P6}{USA TST 2006 P6, That is a prallelogram}