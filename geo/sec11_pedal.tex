\newpage\section{Pedal Triangles}


	\den{Pedal Triangles}{Let $ P $ be an arbitrary point, let $ \triangle A_1B_1C_1 $ be its pedal triangle wrt $ \triangle ABC $. Let $ A', B', C' $ and $ A_0, B_0, C_0 $ be the feet of the altitudes and the midpoints of $ \triangle ABC $. \\
		
		\[ B_1C_1\cap B_0C_0=A_2,\ C_1A_1\cap C_0A_0 = B_2,\ A_1B_1\cap A_0B_0 = C_2 \]
		\[ B'C'\cap B_0C_0=A_3,\ C'A'\cap C_0A_0=B_3,\ A'B'\cap A_0B_0=C_3 \]}
	
	
	
	
	\theo{https://www.awesomemath.org/wp-pdf-files/math-reflections/mr-2015-02/article_1_bocanu.pdf}{Fontene's First Theorem}{$ A_1A_2, B_1B_2, C_1C_2 $ are concurrent at the intersection of $ \odot A_1B_1C_1 $ and $ \odot A_0B_0C_0 $}
	
	
	
	
	\lem{}{$ A'A_3, B'B_3, C'C_3 $ and $ A_0A_3, B_0B_3, C_0C_3 $ concur at the nine point circle of $ \triangle ABC. $}
	
	
	
	\theo{https://www.awesomemath.org/wp-pdf-files/math-reflections/mr-2015-02/article_1_bocanu.pdf}{Fontene's Second Theorem}{Let the concurrency point in the first theorem be $ Q $. Then, if the line $ OP $ is fixed and $ P $ moves along that line, $ Q $ will stay fixed.}
	
	The previous result leads to another beautiful result:
	
	
	
	
	
	\lem{}{Suppose a varying point $ P $ is chosen on the Euler Line of $ \triangle ABC $. Then the pedal circle of $ P $ wrt $ \triangle ABC $ intersects the 9p circle at a fixed point which is the Euler Reflection Point of the median triangle.}
	



		
